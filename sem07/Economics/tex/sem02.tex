\chapter{}
\semdate{19 сентября 2022 г.}

\section{Способы амортизационных отчислений}

\subsection{Линейный способ}

\begin{tcolorbox}
При этом способе \bfit{амортизация начисляется равномерно по годам эксплуатации}:

\begin{equation}
    \text{А}_{\text{г}} = \frac{\text{Ф}_{\text{б}} \cdot
    \text{Н}_{\text{а}}}{100 \%},
\end{equation}

где $\text{А}_{\text{г}}$ --- годовые амортизационные отчисления;

~~~~$\text{Ф}_{\text{б}}$ --- первоначальная (балансовая) стоимость объекта основных
фондов, тыс. руб.;

~~~~~$\text{Н}_{\text{а}}$ --- годовая норма амортизационных
отчислений, лет.
\end{tcolorbox}

\begin{tcolorbox}
В этом случае \bfit{годовая норма амортизации} определяется по формуле:

\begin{equation}
    \text{Н}_{\text{а}} = \frac{100 \%}{\text{Т}_{\text{пи}}},
\end{equation}

где $\text{Т}_{\text{пи}}$ - срок полезного использования объекта основных фондов.
\end{tcolorbox}

Сроком полезного использования является период, обоснованный предприятием как
приносящий прибыль или служащий для выполнения целей организации.

Он определяется нормативными сроками или исходя из:

\begin{itemize}
    \item ожидаемой производительности или мощности применяемого объекта
        основных фондов;
    \item ожидаемого физического износа в соответствии с режимом работы (1-3
        смены), естественных условий, системы планово-предупредительного
        ремонта;
    \item нормативно-правового и другого ограничения использования объекта
        основных фондов (срок аренды).
\end{itemize}

\clearpage
Суть данного способа в том, что каждый год амортизируется равная часть стоимости
данного вида основных средств.

Срок полезного использования объектов определяется организацией самостоятельно
при принятии объекта к бухгалтерскому учету.

\begin{example}
    \normalfont
    ~\\
    Предприятие купило основное средство. Стоимость составила 100~000~руб., срок
    службы --- 5~лет.
\end{example}

\begin{solution}
    \normalfont
    ~\\
    \vspace{-0.5cm}
\begin{enumerate}[parsep=16pt]
    \item $\text{Н}_{\text{а}} = \frac{100\%}{\text{Т}_{\text{пи}}}$
    \item $\text{А}_{\text{г}} = \frac{\text{Ф}_{\text{б}} \cdot
        \text{Н}_{\text{а}}}{100 \%}$
    \item $\text{Н}_{\text{а}} = \frac{100 \%}{5} = 20 \%$
    \item $\text{А}_{\text{г}} = \frac{100 000 \cdot 20 \%}{100 \%} = 20 000$
\end{enumerate}

\begin{tctabularx}{tabularx*={\arrayrulewidth0.5mm}{c|c|c|c},
                   width=0.757\linewidth
    }
    \textbf{Год}
    & \specialcell{ \textbf{Остаточная} \\ \textbf{стоимость на} \\
    \textbf{начало года} \\ \textbf{(руб.)}}
    & \specialcell{ \textbf{Сумма} \\ \textbf{годовой} \\ \textbf{амортизации}
    \\ \textbf{(руб.)}}
    & \specialcell{ \textbf{Остаточная} \\ \textbf{стоимость на} \\
    \textbf{конец года} \\ \textbf{(руб.)}} \\ \hline
    \textbf{1} & 100 000 & 20 000 & 80 000 \\  \hline
    \textbf{2} & 80 000 & 20 000 & 60 000 \\ \hline
    \textbf{3} & 60 000 & 20 000 & 40 000 \\ \hline
    \textbf{4} & 40 000 & 20 000 & 20 000 \\ \hline
    \textbf{5} & 20 000 & 20 000 & 0 \\
\end{tctabularx}
\end{solution}

\subsection{Способ уменьшаемого остатка}

\begin{tcolorbox}

При таком способе \bfit{амортизация} рассчитывается исходя из \bfit{остаточной
стоимости объекта основных фондов на начало отчетного года к нормам
амортизации}:

\begin{equation}
    \text{А}_{i} = (\text{Ф}_{\text{б}i} - \text{A}_{i-1}) \cdot \text{Н}_{\text{а}}
\end{equation}

где $\text{Ф}_{\text{б}i}$ --- остаточная стоимость объекта основных фондов на начало $i$-го года;

~~~~$\text{A}_{i-1}$ --- сумма амортизационных отчислений в $i$-м году;

~~~~$\text{Н}_{\text{а}}$ --- норма амортизации.
\end{tcolorbox}

При использовании данного способа предприятие может применить ускоренную
амортизацию в соответствии с законодательством РФ.

Способ уменьшаемого остатка для определения срока полезного использования
устанавливают в том случае, когда \bfit{эффективность использования объекта
основных средств с каждым последующим годом уменьшается}.

Годовая сумма амортизационных отчислений определяется исходя из остаточной
стоимости объекта основных средств на начало отчетного года и нормы амортизации,
исчисленной исходя из срока полезного использования этого объекта и коэффициента
\bfit{не выше 3, установленного организацией}. Коэффициент устанавливается
организацией самостоятельно, и его величина должна быть отражена в учетной
политике организации.

\begin{example}
    \normalfont
    ~\\
    \textbf{Дано}: остаточная стоимость 260~000~рублей,
    \mbox{$k_{\text{ускорения}}=2$}, срок полезного использования --- 5~лет.
\end{example}

\begin{solution}
    \normalfont
    ~\\
\vspace{-0.5cm}
\begin{enumerate}[parsep=16pt]
    \item $ \text{А}_{i} = (\text{Ф}_{\text{б}i} - \text{A}_{i-1}) \cdot
        \text{Н}_{\text{а}}~[\cdot~k_{\text{ускорения}}] $
    \item $ \text{Н}_{\text{а}} = \frac{100 \%}{5} = 20 \% $
    \item $ \text{Н}_{\text{а}} \cdot k_{\text{ускорения}} = 20 \% \cdot 2 = 40 \% $
    \item $ \text{А}_{1} = 260 000 \cdot 40 \% = 104 000 $
    \item $ \text{А}_{2} = (260 000 - 104 000) \cdot 40 \% = 62 400 $
    \item $ \text{А}_{3} = (156 000 - 62 400) \cdot 40 \% = 37 440 $
    \item $ \text{А}_{4} = (93 600 - 37 440) \cdot 40 \% = 22 464 $
    \item $ \text{А}_{5} = (56 160 - 22 464) \cdot 40 \% = 13 478 $
\end{enumerate}

\begin{tctabularx}{tabularx*={\arrayrulewidth0.5mm}{c|c|c|c},
                   width=0.795\linewidth
    }
    \textbf{Год}
    & \specialcell{ \textbf{Остаточная} \\ \textbf{стоимость на} \\
    \textbf{начало года} \\ \textbf{(руб.)}}
    & \specialcell{ \textbf{Сумма} \\ \textbf{годовой} \\ \textbf{амортизации}
    \\ \textbf{(руб.)}}
    & \specialcell{ \textbf{Остаточная} \\ \textbf{стоимость на} \\ \textbf{конец года} \\ \textbf{(руб.)}} \\ \hline
    \textbf{1} & 260 000 & 104 000 & 156 000 \\  \hline
    \textbf{2} & 156 000 & 62 400 & 93 600 \\ \hline
    \textbf{3} & 93 600 & 37 440 & 56 160 \\ \hline
    \textbf{4} & 56 160 & 22 464 & 33 696 \\ \hline
    \textbf{5} & 33 696 & 13 478 & 20 218 \\ \hline
    \textbf{Итого} &  & 239 782 & \\
\end{tctabularx}

Ликвидационная стоимость --- 20 218 рублей. Если есть условие по последнему
году, то в последний раз считаем от обратного.
\end{solution}

\begin{example}
    \normalfont
    ~\\
    \textbf{Дано:} первоначальная стоимость 128~600~рублей,
    \mbox{$k_{\text{ускорения}} = 2$}, срок полезного использования с января 2012 по
    декабрь 2016  --- 5~лет, ликвидационная стоимость --- 1000~рублей.
\end{example}

\begin{solution}
    \normalfont
    ~\\
    \vspace{-0.5cm}
\begin{enumerate}
    \item $\text{А}_{i} = (\text{Ф}_{\text{б}i} - \text{A}_{i-1}) \cdot \text{Н}_{\text{а}} [\cdot k_{\text{ускорения}}] $
    \item $\text{Н}_{\text{а}} = \frac{100 \%}{5} = 20 \% $
    \item $\text{Н}_{\text{а}} \cdot k_{\text{ускорения}} = 20 \% \cdot 2 = 40 \% $
    \item $\text{А}_{1} = 128 600 \cdot 40 \% = 51 440 $
    \item $\text{А}_{2} = (128 600 - 51 440) \cdot 40 \% = 30 864$
    \item $\text{А}_{3} = (77 160 - 30 864) \cdot 40 \% = 18 519$
    \item $\text{А}_{4} = (46 296 - 18 519) \cdot 40 \% = 11 111$
    \item $\text{А}_{5} = \text{Ф}_{\text{б}5} - \text{ликвидационная стоимость}
        = 16 666 - 1000 = 15 666$
\end{enumerate}

\begin{tctabularx}{tabularx*={\arrayrulewidth0.5mm}{c|c|c|c},
                   width=0.757\linewidth}
    \textbf{Год}
    & \specialcell{ \textbf{Остаточная} \\ \textbf{стоимость на} \\ \textbf{начало года} \\ \textbf{(руб.)}}
    & \specialcell{ \textbf{Сумма} \\ \textbf{годовой} \\ \textbf{амортизации} \\ \textbf{(руб.)}}
    & \specialcell{ \textbf{Остаточная} \\ \textbf{стоимость на} \\ \textbf{конец года} \\ \textbf{(руб.)}} \\ \hline
    \textbf{1} & 128 600 & 51 440 & 77 160 \\  \hline
    \textbf{2} & 77 160 & 30 864 & 46 296 \\ \hline
    \textbf{3} & 46 296 & 18 519 & 27 777 \\ \hline
    \textbf{4} & 27 777 & 11 111 & 16 666 \\ \hline
    \textbf{5} & 16 666 & 15 666 & 1000 \\
\end{tctabularx}
\end{solution}

\subsection{Способ списания стоимости по сумме чисел лет срока полезного
            использования}

\begin{tcolorbox}
При способе списания стоимости по сумме чисел лет срока полезного использования
\bfit{годовая сумма амортизации} определяется исходя из первоначальной стоимости
объекта основных средств и годового соотношения, где в числителе число лет,
остающихся до конца срока службы объекта, а в знаменателе --- сумма чисел лет
срока службы объекта:

\begin{equation}
    \text{А} = \text{Ф}_{\text{б}} \cdot \frac{\text{Т}_{\text{ост}}}{\sum\text{Т}_\text{пи}},
\end{equation}

где $\text{Ф}_{\text{б}}$ --- первоначальная (балансовая) стоимость объекта;

~~~~~$\text{Т}_{\text{ост}}$ --- количество лет, оставшихся до окончания срока
полезного использования;

~~~~~$\sum\text{Т}_\text{пи}$ --- сумма чисел всех лет срока
полезного использования.
\end{tcolorbox}

\begin{example}
    \normalfont
    ~\\
    \textbf{Дано}: остаточная стоимость 260~000~рублей, срок полезного
    использования --- 5~лет.
\end{example}

\begin{solution}
    \normalfont
    ~\\
    \vspace{-0.5cm}
\begin{enumerate}
    \item $ \text{А} = \text{Ф}_{\text{б}} \cdot
        \frac{\text{Т}_{\text{ост}}}{\sum\text{Т}_\text{пи}}, $
    \item $ \sum\text{Т}_\text{пи} = 1 + 2 + 3 + 4 + 5 = 15 $
    \item $ \text{А}_{1} = 260 000 \cdot \frac{5}{15} = 86 667 $
    \item $ \text{А}_{2} = 260 000 \cdot \frac{4}{15} = 69 333 $
    \item $ \text{А}_{3} = 260 000 \cdot \frac{3}{15} = 52 000 $
    \item $ \text{А}_{4} = 260 000 \cdot \frac{2}{15} = 34 667 $
    \item $ \text{А}_{5} = 260 000 \cdot \frac{1}{15} = 17 333 $
\end{enumerate}

\begin{tctabularx}{tabularx*={\arrayrulewidth0.5mm}{c|c|c|c},
                   width=0.794\linewidth}
    \textbf{Год}
    & \specialcell{ \textbf{Остаточная} \\ \textbf{стоимость на} \\ \textbf{начало года} \\ \textbf{(руб.)}}
    & \specialcell{ \textbf{Сумма} \\ \textbf{годовой} \\ \textbf{амортизации} \\ \textbf{(руб.)}}
    & \specialcell{ \textbf{Остаточная} \\ \textbf{стоимость на} \\ \textbf{конец года} \\ \textbf{(руб.)}} \\ \hline
    \textbf{1} & 260 000 & 86 667 & 173 333 \\  \hline
    \textbf{2} & 173 333 & 69 333 & 104 000 \\ \hline
    \textbf{3} & 104 000 & 52 000 & 52 000 \\ \hline
    \textbf{4} & 52 000 & 34 667 & 17 333 \\ \hline
    \textbf{5} & 17 333 & 17 333 & 0 \\ \hline
    \textbf{Итого} &  &  260 000 &  \\
\end{tctabularx}

Ликвидационная стоимость --- 0 рублей.
\end{solution}

\subsection{Способ списания стоимости пропорционально объему продукции (работ)}

\begin{tcolorbox}
При способе списания стоимости основного средства пропорционально объему
продукции (работ, услуг) начисление амортизационных отчислений производится
исходя из натурального показателя объема продукции (работ) в отчетном периоде и
соотношения первоначальной стоимости объекта основных средств и предполагаемого
объема продукции (работ) за весь срок полезного использования объекта основных
средств.

\begin{equation}
    \text{А} = \frac{\text{Ф}_{\text{б}}}{\text{В}},
\end{equation}

где $\text{А}$ --- сумма амортизации на единицу продукции,

~~~~~$\text{Ф}_{\text{б}}$ --- первоначальная стоимость объекта основных
                     средств;

~~~~~$\text{В}$ --- предполагаемый объем производства продукции.
\end{tcolorbox}

Данный метод применяется там, где износ основных средств напрямую связан с
частотой их использования. Чаще всего метод списания стоимости пропорционально
объему продукции используется для расчета амортизации при добыче природного
сырья.

\begin{example}
    \normalfont
    ~\\
    Стоимость автомобиля 65~000~рублей, предполагаемый пробег автомобиля
    400~000~км.
\end{example}

\begin{solution}
    \normalfont
    ~\\
В отчетном периоде пробег автомобиля составил 8 000 км.

$$\text{A} = \frac{\text{стоимость первичная}}{V_{\text{предполагаемый}}} \cdot
V_{\text{фактический}}$$

Сумма амортизации за этот период составит:

$$\text{A} = \frac{65 000}{400 000} \cdot 8 000 = 1 300 \text{ рублей.}$$

Сумма амортизации за весь период составит:

$$\text{A} = \frac{65 000}{400 000} \cdot 400 000 = 65 000 \text{ рублей.}$$
\end{solution}

\begin{example}
    \normalfont
    ~\\
    \textbf{Дано:} первоначальная стоимость 124~000 рублей, срок полезной службы
    с января 2013 по декабрь 2017 --- 5~лет, сумма фактического выпуска ---
    2~480~000~рублей.
\end{example}

\begin{solution}
    \normalfont
    ~\\
    \vspace{-0.5cm}
\begin{enumerate}
    \item $ \text{А} = \frac{\text{Ф}_{\text{б}}}{\text{В}} $
    \item $ \text{А} = \frac{124 000}{2480000} = 0.05 $
    \item $ \text{А}_{1} = 0.05 \cdot 500 000 = 25 000 $
    \item $ \text{А}_{2} = 0.05 \cdot 520 000 = 26 000 $
    \item $ \text{А}_{3} = 0.05 \cdot 460 000 = 23 000 $
    \item $ \text{А}_{4} = 0.05 \cdot 400 000 = 20 000 $
    \item $ \text{А}_{5} = 0.05 \cdot 600 000 = 30 000 $
\end{enumerate}

\begin{tctabularx}{tabularx*={\arrayrulewidth0.5mm}{c|c|c|c|c},
                   width=1.04\linewidth}
    \textbf{Год}
    & \specialcell{ \textbf{Фактический} \\ \textbf{выпуск} \\ \textbf{(руб.)}}
    & \specialcell{ \textbf{Сумма} \\ \textbf{годовой} \\ \textbf{амортизации} \\ \textbf{(руб.)}}
    & \specialcell{ \textbf{Накопленный} \\ \textbf{износ} \\ \textbf{(руб.)}}
    & \specialcell{ \textbf{Остаточная} \\ \textbf{стоимость на} \\
    \textbf{конец года} \\ \textbf{(руб.)}} \\ \hline
    \textbf{1} & 500 000 & 25 000 & 25 000 & 99 000 \\  \hline
    \textbf{2} & 520 000 & 26 000 & 51 000 & 73 000 \\ \hline
    \textbf{3} & 460 000 & 23 000 & 74 000 & 50 000 \\ \hline
    \textbf{4} & 400 000 & 20 000 & 94 000 & 30 000 \\ \hline
    \textbf{5} & 600 000 & 30 000 & 124 000 & 0 \\ \hline
    \textbf{Итого} & 2 480 000 &  &  & \\
\end{tctabularx}
\end{solution}

\section{Вывод}
Проанализировав различные способы начисления амортизации, можно сделать вывод,
что при применении способов уменьшаемого остатка и списания стоимости по сумме
чисел лет срока полезного использования сумма амортизационных отчислений с
годами уменьшается.

Выбирая для начисления амортизации один из этих способов, бухгалтеры должны
помнить о том, что начисленная сумма амортизации влияет на себестоимость
продукции, выполненных работ, оказанных услуг.
