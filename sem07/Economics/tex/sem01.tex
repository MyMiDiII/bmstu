\chapter{}
\semdate{05 сентября 2022 г.}

\section{Классификация имущества предприятия по составу и по источникам
формирования}

Для ведения предпринимательской деятельности предприятие использует
экономические ресурсы: природные, материальные и людские.

Экономические ресурсы могут принимать форму имущества предприятия и капитала.

\begin{definition}
    ~\\
Имущество предприятия - совокупность материальных, нематериальных и финансовых
активов, принадлежащих предприятию.
\end{definition}

В состав материальных активов входят: земельные участки, здания, сооружения,
машины и оборудование, сырье, материалы, комплектующие и полуфабрикаты, готовая
продукция.

Нематериальные активы включают: патенты на изобретения, товарные марки и знаки,
имущественные права, интеллектуальную собственность и нематериальные блага (ст.
128 ГК РФ).

К финансовым активам относят: депозиты в банках, кассовую наличность, расчетные
документы в пути, страховые полисы, вложе ния в ценные бумаги, паи и долевые
вклады в другие предприятия.

Инвестированные в деятельность предприятия экономические ресурсы называют
капиталом.

\section{Видовой состав основных производственных фондов предприятия}

Наименование группы
Состав группы
Назначение и краткая характеристика

1
2
3

Здания

Производственные корпуса, помещения служб, лабораторий, складов, магазинов

Создают комфортные условия для нормального хода производственного процес са,
предохраняют машины и оборудование от воздействия атмосферных осадков

Coopyжения

Шахты, газовые и нефтяные скважины, эстакады, мосты, тоннели, гидротехнические,
водопроводные и канализационные сооружения, путепроводы

Выполняют функции по техническому обслуживанию производства, не связанные с
изменением предмета труда

Передаточные устройства

Устройства электропередачи и связи: электро и теплосеть, трубопроводы, кабельные
линии, воздушные линии связи, канализационные сети, водопроводы

Производят передачу электрической, тепловой и механической энергии от силовых к
рабочим машинам и оборудованию

Машины и оборудование

1) рабочие машины; 2) силовые машины; 3) вычислительная и оргтехника

Непосредственно участвуют в осуществлении производственного процесса, в ходе
которого рабочие воздействуют на предметы труда, превращая их в готовую
продукцию

Транспортные средства

Железнодорожный подвижной состав, средства водного транспорта, автомобили,
воздушный транспорт, средства напольного производственного транспорта

Предназначены для перевозки грузов и людей; функционируют как
внутрипроизводственный и внутрицеховой транспорт

Инструмент

Все виды инструментов для обработки металла, дерева: механический,
пневматический и электрифицированный инструмент

Участвуют в осуществлении производственного процесса и выполняют функции по его
техническому обслуживанию

Измерительные и регулирующие приборы и устройства, лабораторное оборудование

Контрольно-измерительная, проверочная и испытательная аппаратура, пульты
управления, сигнализация и блокировки

Предназначены для автоматизации управления производством, испытания и
лабораторного исследования готовых изделий, исходного сырья, полуфабрикатов и
комплектующих изделий

Инвентарь производственный и хозяйственный

Участвуют в осуществлении производственного процесса (за исключением
хозяйственного инвентаря)

Производственный инвентарь - предметы технического назначения: емкости для
хранения жидкостей, тара, мебель

Хозяйственный инвентарь - предметы конторского и хозяйственного обзаведения,
спортивный инвентарь

Рабочий, продуктивный и племенной скот

Скот, используемый как тягловая сила: скот, производящий продукцию для
купли-продажи

Многолетние насаждения

Насаждения, производящие сырье для пищевой и легкой промышленности

Прочие виды

Включают также земель ные участки и капитальные

