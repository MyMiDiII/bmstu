\chapter{}
\semdate{05 сентября 2022 г.}

%\vspace{-4cm}
\section{Классификация имущества предприятия по составу и по источникам
формирования}


Для ведения предпринимательской деятельности предприятие использует
экономические ресурсы: природные, материальные и людские.

Экономические ресурсы могут принимать форму имущества предприятия и капитала.

Имущество предприятия - совокупность материальных, немате риальных и финансовых активов, принадлежащих предприятию.

В состав материальных активов входят: земельные участки, зда ния, сооружения, машины и оборудование, сырье, материалы, ком плектующие и полуфабрикаты, готовая продукция.

Нематериальные активы включают: патенты на изобретения, то варные марки и знаки, имущественные права, интеллектуальную собственность и нематериальные блага (ст. 128 ГК РФ).

К финансовым активам относят: депозиты в банках, кассовую наличность, расчетные документы в пути, страховые полисы, вложе ния в ценные бумаги, паи и долевые вклады в другие предприятия.

Инвестированные в деятельность предприятия экономические ресурсы называют капиталом.
