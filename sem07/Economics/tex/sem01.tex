\chapter{}
\semdate{05 сентября 2022 г.}

\section{Классификация имущества предприятия по составу и по источникам
формирования}

Для ведения предпринимательской деятельности предприятие использует
экономические ресурсы:
\begin{itemize}
    \item природные,
    \item материальные
    \item и людские.
\end{itemize}

Экономические ресурсы могут принимать форму \bfit{имущества предприятия} и
\bfit{капитала}.

\begin{definition}
    \normalfont
    ~\\
    \bfit{Имущество предприятия} --- совокупность материальных, нематериальных
    и финансовых активов, принадлежащих предприятию.
\end{definition}

В состав \bfit{материальных активов} входят:
\begin{itemize}
    \item земельные участки,
    \item здания,
    \item сооружения,
    \item машины и оборудование,
    \item сырье,
    \item материалы,
    \item комплектующие и полуфабрикаты,
    \item готовая продукция.
\end{itemize}

\bfit{Нематериальные активы} включают (ст. 128 ГК РФ):

\begin{itemize}
    \item патенты на изобретения,
    \item товарные марки и знаки,
    \item имущественные права,
    \item интеллектуальную собственность
    \item и нематериальные блага.
\end{itemize}

~\\

К \bfit{финансовым активам} относят:

\begin{itemize}
    \item депозиты в банках,
    \item кассовую наличность,
    \item расчетные документы в пути,
    \item страховые полисы,
    \item вложения в ценные бумаги,
    \item паи
    \item и долевые вклады в другие предприятия.
\end{itemize}

\begin{definition}
    \normalfont
    ~\\
    Инвестированные в деятельность предприятия экономические ресурсы называют
    \bfit{капиталом}.
\end{definition}

\section{Видовой состав основных производственных фондов предприятия}

\begin{tctabularx}{tabularx*={\arrayrulewidth0.5mm}{m{3.5cm}|m{5.75cm}|m{5.75cm}}}
\thehead{3.5cm}{Наименование группы}
& \thehead{5.75cm}{Состав группы}
& \theheadlast{5.75cm}{Назначение и краткая характеристика}
\\ \specialrule{.1em}{.0em}{.0em}
\thehead{3.5cm}{1} & \thehead{5.75cm}{2} & \theheadlast{5.75cm}{3}
\\ \specialrule{.1em}{.0em}{.0em}
Здания
&
Производственные корпуса, помещения служб, лабораторий, складов, магазинов
&
Создают комфортные условия для нормального хода производственного процесса,
предохраняют машины и оборудование от воздействия атмосферных осадков
\\ \hline
Coopyжения
&
Шахты, газовые и нефтяные скважины, эстакады, мосты, тоннели, гидротехнические,
водопроводные и канализационные сооружения, путепроводы
&
Выполняют функции по техническому обслуживанию производства, не связанные с
изменением предмета труда
\\ \hline
Передаточные устройства
&
Устройства электропередачи и связи: электро и теплосеть, трубопроводы, кабельные
линии, воздушные линии связи, канализационные сети, водопроводы
&
Производят передачу электрической, тепловой и механической энергии от силовых к
рабочим машинам и оборудованию
\end{tctabularx}

\begin{tctabularx}{tabularx*={\arrayrulewidth0.5mm}{m{3.5cm}|m{5.75cm}|m{5.75cm}}}
\thehead{3.5cm}{1} & \thehead{5.75cm}{2} & \theheadlast{5.75cm}{3}
\\ \specialrule{.1em}{.0em}{.0em}
Машины и оборудование
&
1) рабочие машины;\newline2) силовые машины;\newline3) вычислительная
и оргтехника
&
Непосредственно участвуют в осуществлении производственного процесса, в ходе
которого рабочие воздействуют на предметы труда, превращая их в готовую
продукцию
\\ \hline
Транспортные средства
&
Железнодорожный подвижной состав, средства водного транспорта, автомобили,
воздушный транспорт, средства напольного производственного транспорта
&
Предназначены для перевозки грузов и людей; функционируют как
внутрипроизводственный и внутрицеховой транспорт
\\ \hline
Инструмент
&
Все виды инструментов для обработки металла, дерева: механический,
пневматический и электрифицированный инструмент
&
Участвуют в осуществлении производственного процесса и выполняют функции по его
техническому обслуживанию
\\ \hline
Измерительные и регулирующие приборы и устройства, лабораторное оборудование
&
Контрольно-измерительная, проверочная и испытательная аппаратура, пульты
управления, сигнализация и блокировки
&
Предназначены для автоматизации управления производством, испытания и
лабораторного исследования готовых изделий, исходного сырья, полуфабрикатов и
комплектующих изделий
\\ \hline
Инвентарь производственный и хозяйственный
&
Производственный инвентарь --- предметы технического назначения: емкости для
хранения жидкостей, тара, мебель

Хозяйственный инвентарь --- предметы конторского и хозяйственного обзаведения,
спортивный инвентарь
&
Участвуют в осуществлении производственного процесса (за исключением
хозяйственного инвентаря)
\\ \hline
Рабочий, продуктивный и племенной скот
&
Скот, используемый как тягловая сила: скот, производящий продукцию для
купли-продажи
&
\end{tctabularx}

\begin{tctabularx}{tabularx*={\arrayrulewidth0.5mm}{m{3.5cm}|m{5.75cm}|m{5.75cm}}}
\thehead{3.5cm}{1} & \thehead{5.75cm}{2} & \theheadlast{5.75cm}{3}
\\ \specialrule{.1em}{.0em}{.0em}
Многолетние насаждения
&
Насаждения, производящие сырье для пищевой и легкой промышленности
&
\\ \hline
Прочие виды
&
Включают также земельные участки и капитальные
&
\end{tctabularx}

\section{Основные средства}

\begin{definition}
    \normalfont
    ~\\
    \bfit{Основные средства} --- это материально-вещественные ценности (средства
    труда), которые многократно участвуют в производственном процессе, не
    изменяют своей натурально-вещественной формы и переносят свою стоимость на
    готовую продукцию по частям по мере износа.
\end{definition}

С точки зрения учета и оценки основные средства представляют собой часть
имущества, которая многократно используется в качестве средств труда, при
производстве продукции, выполнении работ и оказании услуг или для управления
организаций в течение периода, который превышает 12 месяцев.

Отличительной особенностью основных средств является их многократное
использование в процессе производства, сохранение первоначального внешнего вида
(формы) в течение длительного периода.

\section{Структура основных производственных фондов}

\section{Износ основных фондов}

Два основных метода определения \bfit{степени физического износа}:

\begin{enumerate}
    \item \textit{По техническому состоянию}, исходя из экспертной оценки объекта.
    \item \textit{По срокам службы или по объемам работы}.
\end{enumerate}

\begin{tcolorbox}
\bfit{Коэффициент физического износа} отдельных видов основных фондов по сроку
службы определяется по формуле:

\begin{equation}
    K_{\text{ф.н.}} = T_{\text{ф}}/T_{\text{н}},
\end{equation}

где $T_{\text{ф}}$ --- фактический срок службы, лет;

~~~~~$T_{\text{н}}$ --- нормативный срок службы (амортизационный период), лет.
\end{tcolorbox}

\begin{tcolorbox}
Величина \bfit{морального износа} определяется по формуле:

\begin{equation}
    \text{И}_{\text{ф}} = \text{Ф}_{\text{пс}} - \text{Ф}_{\text{пс}} \cdot
    \frac{W_{\text{ст}} \cdot T_{\text{ст}}}{W_{\text{н}} \cdot T_{\text{н}}}
\end{equation}

где $\text{Ф}_{\text{пс}}$, $\text{Ф}_{\text{пс}}$ --- соответственно
первоначальная стоимость морально устаревшего (старого) и нового
оборудования,~руб.;

~~~~$W_{\text{ст}}$, $W_{\text{н}}$ --- годовая производительность морально
устаревшего и нового оборудования;

~~~~~$T_{\text{ст}}$, $T_{\text{н}}$ --- срок службы морально устаревшего и нового
оборудования,~лет.
\end{tcolorbox}

\begin{definition}
    \normalfont
    ~\\
    \bfit{Износ основных фондов} --- частичная или полная утрата ими
    потребительских свойств и стоимости.
\end{definition}

\begin{definition}
    \normalfont
    ~\\
    \bfit{Срок полезного использования основных фондов} --- период времени, в
    течение которого объект основных фондов призван приносить доход предприятию
    и служить основным целям его деятельности.
\end{definition}

\imgs{fonds}{h!}{1}{Износ основных фондов}

\section{Оценка основных средств}

Существуют 3 метода оценки основных средств:

\begin{enumerate}
    \item \textit{По первоначальной стоимости}

\begin{definition}
    \normalfont
    ~\\
    \bfit{Первоначальная стоимость} --- это сумма фактических затрат организации
    на приобретение, доставку и доведение до рабочего состояния основных
    средств.
    ~\\
    \bfit{Первоначальная стоимость} --- это фактическая стоимость создания основных средств.
\end{definition}

По первоначальной стоимости основные средства учитываются и оцениваются в ценах
тех лет, когда они были созданы.

\item \textit{По восстановительной стоимости}

\begin{definition}
    \normalfont
    ~\\
    \bfit{Восстановительная стоимость} --- это стоимость воспроизводства
    основных средств в современных конкретных эксплуатационных условиях.
\end{definition}

Восстановительная стоимость показывает, сколько денежных средств пришлось бы
затратить предприятию в данный момент времени для замены имеющихся изношенных в
той или иной степени основных средств такими же, но новыми.

Восстановительная стоимость определяется путем переоценки основных средств.

В настоящее время предприятие имеет право самостоятельно не чаще 1 раза в год
(на начало отчетного периода) производить переоценку основных средств.

3. \textit{По остаточной стоимости}

\begin{definition}
    \normalfont
    ~\\
    \bfit{Остаточная стоимость} --- это стоимость, еще не перенесенная на
    готовую продукцию.
\end{definition}

Остаточная стоимость определяется как \textit{разность между первоначальной
(восстановительной) стоимостью и суммой начисленной амортизации}.

Основные средства учитываются на предприятии по первоначальной стоимости, а
после переоценки --- по восстановительной стоимости.

В балансе предприятия основные средства отражаются по остаточной стоимости.
\end{enumerate}

Кроме этого, можно выделить два вида оценки основных средств:

\begin{itemize}
    \item ликвидационная стоимость
        \begin{definition}
            \normalfont
            ~\\
            \bfit{Ликвидационная стоимость} --- это стоимость возможной
            реализации выбывающих, полностью изношенных основных средств.
        \end{definition}

    \item амортизируемая стоимость
        \begin{definition}
            \normalfont
            ~\\
            \bfit{Амортизируемая стоимость} --- это стоимость, которую
            необходимо перенести на готовую продукцию.
        \end{definition}
\end{itemize}

В российской экономической практике это первоначальная
(восстановительная) стоимость, в мировой практике --- разность между
первоначальной и ликвидационной стоимостью.

\section{Амортизация}

\section{Способы амортизационных отчислений}
