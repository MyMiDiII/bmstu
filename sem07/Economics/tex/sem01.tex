\chapter{}
\semdate{05 сентября 2022 г.}

\section{Классификация имущества предприятия по составу и по источникам
формирования}

Для ведения предпринимательской деятельности предприятие использует
экономические ресурсы:
\begin{itemize}
    \item природные,
    \item материальные
    \item и людские.
\end{itemize}

Экономические ресурсы могут принимать форму \bfit{имущества предприятия} и
\bfit{капитала}.

\begin{definition}
    \normalfont
    ~\\
    \bfit{Имущество предприятия} --- совокупность материальных, нематериальных
    и финансовых активов, принадлежащих предприятию.
\end{definition}

В состав \bfit{материальных активов} входят:
\begin{itemize}
    \item земельные участки,
    \item здания,
    \item сооружения,
    \item машины и оборудование,
    \item сырье,
    \item материалы,
    \item комплектующие и полуфабрикаты,
    \item готовая продукция.
\end{itemize}

\bfit{Нематериальные активы} включают (ст. 128 ГК РФ):

\begin{itemize}
    \item патенты на изобретения,
    \item товарные марки и знаки,
    \item имущественные права,
    \item интеллектуальную собственность
    \item и нематериальные блага.
\end{itemize}

~\\

К \bfit{финансовым активам} относят:

\begin{itemize}
    \item депозиты в банках,
    \item кассовую наличность,
    \item расчетные документы в пути,
    \item страховые полисы,
    \item вложения в ценные бумаги,
    \item паи
    \item и долевые вклады в другие предприятия.
\end{itemize}

\begin{definition}
    \normalfont
    ~\\
    Инвестированные в деятельность предприятия экономические ресурсы называют
    \bfit{капиталом}.
\end{definition}

\section{Видовой состав основных производственных фондов предприятия}

\begin{tctabularx}{tabularx*={\arrayrulewidth0.5mm}{p{3.5cm}|p{5.75cm}|p{5.75cm}}}
\thehead{3.5cm}{Наименование группы}
& \thehead{5.75cm}{Состав группы}
& \theheadlast{5.75cm}{Назначение и краткая характеристика}
\\ \specialrule{.1em}{.0em}{.0em}
\thehead{3.5cm}{1} & \thehead{5.75cm}{2} & \theheadlast{5.75cm}{3}
\\ \specialrule{.1em}{.0em}{.0em}
Здания
&
Производственные корпуса, помещения служб, лабораторий, складов, магазинов
&
Создают комфортные условия для нормального хода производственного процесса,
предохраняют машины и оборудование от воздействия атмосферных осадков
\\ \hline
Coopyжения
&
Шахты, газовые и нефтяные скважины, эстакады, мосты, тоннели, гидротехнические,
водопроводные и канализационные сооружения, путепроводы
&
Выполняют функции по техническому обслуживанию производства, не связанные с
изменением предмета труда
\\ \hline
Передаточные устройства
&
Устройства электропередачи и связи: электро и теплосеть, трубопроводы, кабельные
линии, воздушные линии связи, канализационные сети, водопроводы
&
Производят передачу электрической, тепловой и механической энергии от силовых к
рабочим машинам и оборудованию
\end{tctabularx}

\begin{tctabularx}{tabularx*={\arrayrulewidth0.5mm}{p{3.5cm}|p{5.75cm}|p{5.75cm}}}
\thehead{3.5cm}{1} & \thehead{5.75cm}{2} & \theheadlast{5.75cm}{3}
\\ \specialrule{.1em}{.0em}{.0em}
Машины и оборудование
&
1) рабочие машины;\newline2) силовые машины;\newline3) вычислительная
и оргтехника
&
Непосредственно участвуют в осуществлении производственного процесса, в ходе
которого рабочие воздействуют на предметы труда, превращая их в готовую
продукцию
\\ \hline
Транспортные средства
&
Железнодорожный подвижной состав, средства водного транспорта, автомобили,
воздушный транспорт, средства напольного производственного транспорта
&
Предназначены для перевозки грузов и людей; функционируют как
внутрипроизводственный и внутрицеховой транспорт
\\ \hline
Инструмент
&
Все виды инструментов для обработки металла, дерева: механический,
пневматический и электрифицированный инструмент
&
Участвуют в осуществлении производственного процесса и выполняют функции по его
техническому обслуживанию
\\ \hline
Измерительные и регулирующие приборы и устройства, лабораторное оборудование
&
Контрольно-измерительная, проверочная и испытательная аппаратура, пульты
управления, сигнализация и блокировки
&
Предназначены для автоматизации управления производством, испытания и
лабораторного исследования готовых изделий, исходного сырья, полуфабрикатов и
комплектующих изделий
\\ \hline
Инвентарь производственный и хозяйственный
&
Производственный инвентарь --- предметы технического назначения: емкости для
хранения жидкостей, тара, мебель

Хозяйственный инвентарь --- предметы конторского и хозяйственного обзаведения,
спортивный инвентарь
&
Участвуют в осуществлении производственного процесса (за исключением
хозяйственного инвентаря)
\\ \hline
Рабочий, продуктивный и племенной скот
&
Скот, используемый как тягловая сила: скот, производящий продукцию для
купли-продажи
&
\end{tctabularx}

\begin{tctabularx}{tabularx*={\arrayrulewidth0.5mm}{p{3.5cm}|p{5.75cm}|p{5.75cm}}}
\thehead{3.5cm}{1} & \thehead{5.75cm}{2} & \theheadlast{5.75cm}{3}
\\ \specialrule{.1em}{.0em}{.0em}
Многолетние насаждения
&
Насаждения, производящие сырье для пищевой и легкой промышленности
&
\\ \hline
Прочие виды
&
Включают также земельные участки и капитальные
&
\end{tctabularx}

\section{Основные средства}

\begin{definition}
    \normalfont
    ~\\
    \bfit{Основные средства} --- это материально-вещественные ценности (средства
    труда), которые многократно участвуют в производственном процессе, не
    изменяют своей натурально-вещественной формы и переносят свою стоимость на
    готовую продукцию по частям по мере износа.
\end{definition}

С точки зрения учета и оценки основные средства представляют собой часть
имущества, которая многократно используется в качестве средств труда, при
производстве продукции, выполнении работ и оказании услуг или для управления
организаций в течение периода, который превышает 12 месяцев.

Отличительной особенностью основных средств является их многократное
использование в процессе производства, сохранение первоначального внешнего вида
(формы) в течение длительного периода.

\section{Структура основных производственных фондов}

Соотношение между отдельными группами и частями основных производственных фондов
характеризует их структуру, имеющую важное значение в организации производства.
Наиболее эффективна та структура, где больше удельный вес активной части.

В отраслях экономики структура основных производственных фондов неодинакова.

\bfit{Важнейшими факторами}, влияющими на структуру основных производственных
фондов предприятия, являются:

\begin{itemize}
	\item характер выпускаемой продукции;
	\item объем выпуска продукции;
	\item уровень механизации и автоматизации;
	\item уровень специализации и кооперирования;
	\item климатические и географические условия расположения предприятий.
\end{itemize}

\begin{table}[h]
    \captionsetup{format=hang,justification=centering,
                  singlelinecheck=off,width=0.907\linewidth,
                  labelformat=empty}
    \begin{center}
        \caption{\textbf{Состав и структура основных производственных
                 фондов предприятий промышленности и машиностроения}}
        \vspace{-0.3cm}
        \begin{tctabularx}{tabularx*={\arrayrulewidth0.5mm}{l|l|l},
                           width=0.907\linewidth,
                           }
            & \multicolumn{2}{c|}{\textbf{В процентах к итогу}} \\
            \cline{2-3}
            \cellcolor{black!7!white}\raisebox{1ex}[0cm][0cm]{\specialcell{
            \textbf{Основные} \\ \textbf{производственные фонды} }}
            & \textbf{промышленность}
            & \textbf{машиностроение} \\ \hline
            Здания & 28.9 & 39.5 \\
            Сооружения & 18.8 & 7.1 \\
            Передаточные устройства & 11.3 & 4.0 \\
            Машины и оборудование & 37.7 & 44.9 \\
            Транспортные средства & 2.2 & 2.4 \\
            Прочие основные фонды & 1.1 & 2.1 \\
            \textbf{Итого} & 100.0 & 100.0
        \end{tctabularx}
    \end{center}
    \label{tab:fond-struct}
\end{table}

\section{Учет основных производственных фондов}

Основные средства учитываются в \bfit{натуральных} и \bfit{стоимостных}
показателях.

\bfit{Натуральные} необходимы для

\begin{itemize}
    \item установления количества и состава основных фондов,
    \item расчета производственной мощности,
    \item организации ремонта
    \item и замены оборудования.
\end{itemize}

\bfit{Стоимостные показатели} необходимы для
\begin{itemize}
    \item определения общей стоимости структуры и динамики основных средств,
    \item расчета амортизационных отчислений, себестоимости,
        рентабельности~и~т.~д.
\end{itemize}

Основные средства оценивают по~(см.~подраздел~\ref{evaluation})

\begin{itemize}
    \item первоначальной,
    \item восстановительной,
    \item  остаточной,
    \item ликвидационной,
    \item среднегодовой стоимости.
\end{itemize}

\begin{important}
    Основные средства принимаются к бухгалтерскому учету по первоначальной
    стоимости.
\end{important}

Основные средства принимаются к учету в случае:

\begin{itemize}
	\item их приобретения, сооружения или изготовления;
	\item внесения учредителями в счет их вклада в уставный капитал;
	\item получения по договору дарения и безвозмездного получения.
\end{itemize}

\section{\label{evaluation}Оценка основных средств}

Существуют 3 метода оценки основных средств:

\begin{enumerate}
    \item \textit{По первоначальной стоимости}
    \item \textit{По восстановительной стоимости}
    \item \textit{По остаточной стоимости}
\end{enumerate}

\begin{definition}
    \normalfont
    ~\\
    \bfit{Первоначальная стоимость} --- это сумма фактических затрат организации
    на приобретение, доставку и доведение до рабочего состояния основных
    средств.
    ~\\
    \bfit{Первоначальная стоимость} --- это фактическая стоимость создания основных средств.
\end{definition}

По первоначальной стоимости основные средства учитываются и оцениваются в ценах
тех лет, когда они были созданы.


\begin{definition}
    \normalfont
    ~\\
    \bfit{Восстановительная стоимость} --- это стоимость воспроизводства
    основных средств в современных конкретных эксплуатационных условиях.
\end{definition}

Восстановительная стоимость показывает, сколько денежных средств пришлось бы
затратить предприятию в данный момент времени для замены имеющихся изношенных в
той или иной степени основных средств такими же, но новыми.

Восстановительная стоимость определяется путем переоценки основных средств.

В настоящее время предприятие имеет право самостоятельно не чаще 1 раза в год
(на начало отчетного периода) производить переоценку основных средств.


\begin{definition}
    \normalfont
    ~\\
    \bfit{Остаточная стоимость} --- это стоимость, еще не перенесенная на
    готовую продукцию.
\end{definition}

Остаточная стоимость определяется как \textit{разность между первоначальной
(восстановительной) стоимостью и суммой начисленной амортизации}.

Основные средства учитываются на предприятии по первоначальной стоимости, а
после переоценки --- по восстановительной стоимости.

В балансе предприятия основные средства отражаются по остаточной стоимости.

Кроме этого, можно выделить два вида оценки основных средств:

\begin{itemize}
    \item ликвидационная стоимость,
    \item амортизируемая стоимость.
\end{itemize}

\begin{definition}
    \normalfont
    ~\\
    \bfit{Ликвидационная стоимость} --- это стоимость возможной
    реализации выбывающих, полностью изношенных основных средств.
\end{definition}

\begin{definition}
    \normalfont
    ~\\
    \bfit{Амортизируемая стоимость} --- это стоимость, которую
    необходимо перенести на готовую продукцию.
\end{definition}

В российской экономической практике это первоначальная
(восстановительная) стоимость, в мировой практике --- разность между
первоначальной и ликвидационной стоимостью.

\clearpage
\section{Износ основных фондов}

Два основных метода определения \bfit{степени физического износа}:

\begin{enumerate}
    \item \textit{По техническому состоянию}, исходя из экспертной оценки объекта.
    \item \textit{По срокам службы или по объемам работы}.
\end{enumerate}

\begin{tcolorbox}
\bfit{Коэффициент физического износа} отдельных видов основных фондов по сроку
службы определяется по формуле:

\begin{equation}
    K_{\text{ф.н.}} = \frac{T_{\text{ф}}}{T_{\text{н}}},
\end{equation}

где $T_{\text{ф}}$ --- фактический срок службы, лет;

~~~~~$T_{\text{н}}$ --- нормативный срок службы (амортизационный период), лет.
\end{tcolorbox}

\begin{tcolorbox}
Величина \bfit{морального износа} определяется по формуле:

\begin{equation}
    \text{И}_{\text{ф}} = \text{Ф}_{\text{пс}} - \text{Ф}_{\text{пс}} \cdot
    \frac{W_{\text{ст}} \cdot T_{\text{ст}}}{W_{\text{н}} \cdot T_{\text{н}}}
\end{equation}

где $\text{Ф}_{\text{пс}}$, $\text{Ф}_{\text{пс}}$ --- соответственно
первоначальная стоимость морально устаревшего (старого) и нового
оборудования,~руб.;

~~~~$W_{\text{ст}}$, $W_{\text{н}}$ --- годовая производительность морально
устаревшего и нового оборудования;

~~~~~$T_{\text{ст}}$, $T_{\text{н}}$ --- срок службы морально устаревшего и нового
оборудования,~лет.
\end{tcolorbox}

\begin{definition}
    \normalfont
    ~\\
    \bfit{Износ основных фондов} --- частичная или полная утрата ими
    потребительских свойств и стоимости.
\end{definition}

\begin{definition}
    \normalfont
    ~\\
    \bfit{Срок полезного использования основных фондов} --- период времени, в
    течение которого объект основных фондов призван приносить доход предприятию
    и служить основным целям его деятельности.
\end{definition}

\imgs{fonds}{h!}{0.95}{Износ основных фондов}

\section{Амортизация}

Возмещение износа основных фондов осуществляется на основе амортизации.

\begin{definition}
    \normalfont
    ~\\
    \bfit{Амортизация} --- это процесс переноса стоимости основных средств на
    готовую продукцию и возмещение этой стоимости в процессе реализации
    продукции.
\end{definition}

\begin{definition}
    \normalfont
    ~\\
    \bfit{Амортизация} --- это процесс постепенного перенесения стоимости
    основных фондов на производимую продукцию в целях накопления средств для
    полного их восстановления (реновации).
\end{definition}

При исчислении амортизации предприятие самостоятельно определяет норму
амортизации и метод амортизации, при этом основную роль играет \bfit{срок
полезного использования} основных средств --- это период, в течение которого
использование объекта основных средств призвано приносить доход или служить для
выполнения целей деятельности организации.

Денежным выражением размера амортизации являются
\bfit{амортизационные отчисления}, и 10\%, начисленные на издержки
производства (себестоимость) на основе нормы амортизации.

\begin{definition}
    \normalfont
    ~\\
    \bfit{Амортизационные отчисления} --- это денежное выражение размера
    амортизации, которое должно соответствовать степени износа основных средств.
\end{definition}

Амортизационные отчисления включаются в себестоимость продукции.

Способы амортизационных отчислений:
\begin{itemize}
	\item линейный;
	\item уменьшаемого остатка;
	\item списания стоимости по сумме чисел лет срока полезного использования;
	\item списания стоимости пропорционально объему продукции (работ).
\end{itemize}

Амортизационные группы амортизируемого имущества представлены в таблице ниже.

\begin{table}[h]
    \captionsetup{format=hang,justification=centering,
                  singlelinecheck=off,width=0.907\linewidth,
                  labelformat=empty}
    \caption{\textbf{Состав и структура основных производственных
             фондов предприятий промышленности и машиностроения}}
\begin{tctabularx}{tabularx*={\arrayrulewidth0.5mm}{p{5cm}|p{5cm}|p{5cm}}}
    \thehead{5cm}{Амортизационная группа}
    & \thehead{5cm}{Срок полезного использования имущества, лет}
    & \theheadlast{5cm}{Метод расчета суммы амортизации} \\ \hline
    I & 1-2 & Линейный или \\
    II & 2-3 & нелинейный метод \\
    III & 3-5 & (по выбору) \\
    IV & 5-7 & \\
    V & 7-10 & \\
    VI & 10-15 & \\
    VII & 15-20 & \\ \hline
    VIII & 20-25 & Линейный метод \\
    IX & 25-30 &  \\
    X & Свыше 30 & \\
\end{tctabularx}
\end{table}
