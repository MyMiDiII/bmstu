\chapter{Выполнение домашнего задания}

\section{Определение выбранной продукции}

Для выполнения домашнего задания, будет рассмотрено предприятие, которое планирует выпускать механические клавиатуры.

\textbf{Механическая клавиатура} --- клавиатура, у которой за работу каждой кнопки отвечают отдельные переключатели с особым механизмом. Прородителем механической клавиатуры считается пишущая машинка.

Контакты внутри переключателей размыкаются механически, а за счёт металлической пружины кнопка мягко возвращается в исходное положение. При этом свитчи быстро реагируют и срабатывают раньше, чем клавиша полностью <<продавится>>. Момент активации хорошо ощущается тактильно, что даёт лучший контроль и требует меньше усилий при наборе.


\section{Комплекс работ}

В таблице \ref{tbl:works} представлен комплекс работ для производства механических клавиатур.

% Настройка выравнивание колонки по центру
\newcolumntype{P}[1]{>{\centering\arraybackslash}p{#1}}

\begin{center}
    \captionsetup{justification=raggedright,singlelinecheck=off}
    \begin{longtable}[c]{|P{2cm}|P{2cm}|P{5cm}|P{6cm}|}
    \caption{Комплекс работ\label{tbl:works}}
    \\ \hline
        \textbf{Номер события} & 
        \textbf{Цифры работ} & 
        \textbf{Продолжительность работ (недель)} & 
        \textbf{Наименование и содержание работ} 
    \\ \hline
        1 &
        --- &
        0 &
        ---    
    \\ \hline
        2 &
        1-2 &
        4 &
        Заключение договора с продавцом материалов
    \\ \hline
        3 &
        1-3 &
        3 &
        Разработка дизайна
    \\ \hline
        4 &
        2-4 &
        2 &
        Закупка материалов
    \\ \hline
        5 &
        4-5 &
        1 &
        Покраска материалов
    \\ \hline
        6 &
        5-6 &
        2 &
        Сборка основной платы
    \\ \hline
        7 &
        6-7 &
        2 &
        Программирование основной платы
    \\ \hline
        8 &
        7-8 &
        1 &
        Печать символов на клавишах
    \\ \hline
        9 &
        4-9 &
        9 &
        Полная сборка
    \\ \hline
        9 &
        8-9 &
        2 &
        Сборка корпуса
    \\ \hline
        10 &
        9-10 &
        5 &
        Тестирование клавиатуры
    \\ \hline
        11 &
        3-11 &
        10 &
        Заключение договора с дистрибьютором
    \\ \hline
        11 &
        10-11 &
        2 &
        Представление результата дистрибьютору
    \\ \hline
        12 &
        10-12 &
        3 &
        Упаковка
    \\ \hline
        12 &
        11-12 &
        3 &
        Отправка дистрибьютору
    \\ \hline
\end{longtable}
\end{center}


\section{Построение сетевого графика}

Определение ранних начал работ представлены в таблице \ref{tbl:work-begin}. Необходимо учесть, что если в одно событие идет несколько событий, то выбирается \textbf{наибольшее} количество недель.

\begin{center}
    \captionsetup{justification=raggedright,singlelinecheck=off}
    \begin{longtable}[c]{|P{2cm}|P{7cm}|}
    \caption{Ранние начала работ\label{tbl:work-begin}}
    \\ \hline
        \textbf{События} & \textbf{Недель}
    \\ \hline
        1--2 & 0 + 4 = 4
    \\ \hline
        1--3 & 0 + 3 = 3
    \\ \hline
        2--4 & 4 + 2 = 6
    \\ \hline
        4--5 & 6 + 1 = 7
    \\ \hline
        5--6 & 7 + 2 = 9
    \\ \hline
        6--7 & 9 + 2 = 11
    \\ \hline
        7--8 & 11 + 1 = 12
    \\ \hline
        4--9 & 6 + 9 = 15 (max) \\
        8--9 & 12 + 2 = 14
    \\ \hline
        9--10 & 15 + 5 = 20
    \\ \hline
        3--11 & 3 + 10 = 13 (max) \\
        10--11 & 20 + 2 = 22 
    \\ \hline
        10--12 & 20 + 2 = 22 \\
        11--12 & 22 + 3 = 25 (max)
    \\ \hline
\end{longtable}
\end{center}


А в таблице \ref{tbl:work-end} представлено определение поздних начал работ. При этом нужно учесть, что если в одно событие идет несколько событий, то выбирается \textbf{наименьшее} количество недель.

\begin{center}
    \captionsetup{justification=raggedright,singlelinecheck=off}
    \begin{longtable}[c]{|P{2cm}|P{7cm}|}
    \caption{Поздние начала работ\label{tbl:work-end}}
    \\ \hline
        \textbf{События} & \textbf{Недель}
    \\ \hline
        12--11 & 25 - 3 = 22
    \\ \hline
        11--10 & 22 - 2 = 20 (min) \\
        12--10 & 25 - 3 = 22
    \\ \hline
        10--9 & 20 - 5 = 15
    \\ \hline
        9--8 & 15 - 2 = 13
    \\ \hline
        8--7 & 13 - 1 = 12
    \\ \hline
        7--6 & 12 - 2 = 10
    \\ \hline
        6--5 & 10 - 2 = 8
    \\ \hline
        9--4 & 15 - 9 = 6 (min) \\
        5--4 & 8 - 1 = 7
    \\ \hline
        11--3 & 22 - 10 = 12
    \\ \hline
        3--1 & 12 - 3 = 9 \\
        2--1 & 4 - 4 = 0 (min)
    \\ \hline
\end{longtable}
\end{center}

Таким образом, на рисунке \ref{img:economic-scheme} представлен сетевой график выполнения работ. На приведенном рисунке также были вычислены резервы (как разность раннего и позднего начал работ каждого отдельно взятого события).

Также из рисунка \ref{img:economic-scheme} видно, что \textbf{критическим путем} является следующая последовательность событий:

\begin{equation}
    1 --- 2 --- 4 --- 9 --- 10 --- 11 --- 12
\end{equation}

При этом его \textbf{продолжительность} равна $25$ неделям. 

%\imgs{economic-scheme}{h!}{0.75}{Сетевой график выполнения комплекса работ}


\section{Прибыль предприятия}

\subsection{Номенклатура переменных и постоянных затрат}

Номенклатура переменных затрат представлена в таблице \ref{tbl:var}, а постоянных --- в таблице \ref{tbl:const}.

\begin{center}
    \captionsetup{justification=raggedright,singlelinecheck=off}
    \begin{longtable}[c]{|P{6cm}|P{7cm}|}
    \caption{Переменные затраты\label{tbl:var}}
    \\ \hline
        \textbf{Наименование} & \textbf{Рублей на единицу продукции}
    \\ \hline
        Материалы & $8\ 000$
    \\ \hline
        Электричество & $300$
    \\ \hline
        Упаковка & $500$
    \\ \hline
        Кабель зарядки & $500$
    \\ \hline
        Транспортировка & $200$
    \\ \hline
        \textbf{Итого} & $\textbf{9 000}$
    \\ \hline
\end{longtable}
\end{center}


\begin{center}
    \captionsetup{justification=raggedright,singlelinecheck=off}
    \begin{longtable}[c]{|P{6cm}|P{7cm}|}
    \caption{Постоянные затраты\label{tbl:const}}
    \\ \hline
        \textbf{Наименование} & \textbf{Млрд руб в год}
    \\ \hline
        Заработная плата & $1.3$
    \\ \hline
        Аренда помещений & $0.6$
    \\ \hline
        Коммунальные расходы & $0.3$
    \\ \hline
        Реклама & $0.4$
    \\ \hline
        Амортизация & $0.6$
    \\ \hline
        \textbf{Итого} & $\textbf{3.2}$
    \\ \hline
\end{longtable}
\end{center}


\subsection{Рассчет прибыли предприятия}

Средняя цена реализации продукта равно $12\ 000$ руб, а планируемый объем производства в год --- $6\ 000\ 000$ штук. Тогда:

\begin{equation}
    \textbf{Выручка от реализации} = 12\ 000 \cdot 6\ 000\ 000 = 72 \text{ млрд руб.}
\end{equation}

Стоимость переменных затрат на единицу продукции равна $9\ 000$ руб, тогда для всего объема производства в год:

\begin{equation}
    \textbf{Переменные затраты} = 9\ 000 \cdot 6\ 000\ 000 = 54 \text{ млрд руб.}
\end{equation}

Таким образом, маржинальный доход будет вычислен как разность выручки от реализации и переменных затрат, то есть

\begin{equation}
    \textbf{Маржинальный доход} = 72 - 54 = 18 \text{ млрд руб.}
\end{equation}

Поскольку постоянные расходы равны $3.2$ млрд руб в год, то годовая прибыль будет вычислена как разность маржинального дохода и постоянных расходов, то есть

\begin{equation}
    \textbf{Годовой доход} = 18 - 3.2 = 14.8 \text{ млрд руб.}
\end{equation}

Зная годовой доход и планируемый объем производства в год, можно вычислить среднюю величину маржинального дохода на единицу продукции:

\begin{equation}
    \textbf{Средняя величина маржинального дохода} = \frac{14\ 800\ 000\ 000}{6\ 000\ 000}  = 2\ 467 \text{ руб.}
\end{equation}

Также найдем точку безубыточности. В точке безубыточности прибыль равна нулю, поэтому эта точка может быть найдена при условии равенства выручки и суммы переменных и постоянных затрат, то есть получается следующее уравнение, где $X$ --- точка безубыточности:

\begin{equation}
    \begin{gathered}
        12\ 000 \cdot X = 9\ 000 \cdot X + 3\ 200\ 000\ 000 
        \\ 3\ 000 \cdot X = 3\ 200\ 000\ 000 
        \\ X = \frac{3\ 200\ 000\ 000}{3\ 000} = 1\ 066\ 067 \text{  штук } = 1.07 \text{  млн штук}
    \end{gathered}
\end{equation}


Таким образом, все результаты вычислений представлены в обощающей таблице \ref{tbl:result} прибыли предприятия при установленном объеме реализации продукции.

\begin{center}
    \captionsetup{justification=raggedright,singlelinecheck=off}
    \begin{longtable}[c]{|P{2cm}||P{10cm}|P{2cm}|}
    \caption{Прибыль предприятия\label{tbl:result}}
    \\ \hline
        \textbf{№ п/п} & \textbf{Показатели} & \textbf{Значение}
    \\ \hline
        1 & Объем производства, млн штук & $6$
    \\ \hline
        2 & Выручка от реализации, млрд руб & $72$
    \\ \hline
        3 & Перенные затраты, млрд руб & $54$
    \\ \hline
        4 & Маржинальный доход, млрд руб & $18$
    \\ \hline
        5 & Постоянные затраты, млрд руб & $3.2$
    \\ \hline
        6 & Прибыль, млрд руб & $14.8$
    \\ \hline
        7 & Средняя величина маржинального дохода, руб & $2\ 467$
    \\ \hline
\end{longtable}
\end{center}





















% \captionsetup{justification=raggedleft,singlelinecheck=off}
% \begin{table}[H]
%     \centering
% 	\caption{Характеристики варианта}
%     \label{tbl:var19}
% 	\begin{tabular}{|c|c|c|c|c|c|}
%         \hline
%         \multicolumn{2}{|c|}{Характеристика помещения} & 
%         \multicolumn{4}{|c|}{Характеристиказрительных работ} \\ 
%         \hline
%         Тип &
%         $A \times B$, м & 
%         \multicolumn{3}{|c|}{Рассматриваемый объект} & 
%         \begin{minipage}[t]{1.5cm}\centering Цвет \newline фона\end{minipage}
%         \\ \cline{3-5}
%         & 
%         & 
%         \begin{minipage}[t]{2cm}\centering Вид\newline работ\end{minipage} & 
%         \begin{minipage}[t]{3cm} \centering Размер\newline объекта, мм\end{minipage} &
%         \begin{minipage}[t]{2cm} \centeringЦвет\newline объекта\end{minipage} &
%         \\ \hline
%         \begin{minipage}[t]{2.5cm}\centering Кабинет\newline бухгалтера\end{minipage} &
%         $5 \times 3$ &
%         \begin{minipage}[t]{2.5cm}\centering Платежные\newline чеки\end{minipage} &
%         0.5 &
%         Серый &
%         Серый
%         \\ \hline
%     \end{tabular}
% \end{table}
