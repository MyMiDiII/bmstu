\chapter{Определение выбранной продукции}

Для выполнения домашнего задания рассмотривается предприятие, которое планирует
выпускать чехлы для телефонов.

\bfit{Чехол для телефона} --- аксессуар устройства связи, выполняющий защитные и
декоративные функции. Иными словами чехол защищает смартфон от случайных
повреждений, например, царапин, падений и попадания влаги, а возможность
нанесения на чехол уникального рисунка позволяет сделать свой смартфон
уникальным.

\chapter{Комплекс работ}

В таблице~\ref{tbl:works} представлен комплекс работ для производства чехлов для
телефона.

% Настройка выравнивание колонки по центру
\newcolumntype{P}[1]{>{\centering\arraybackslash}p{#1}}

\vspace{-0.2cm}
{
\captionsetup{format=hang,justification=raggedright,
              singlelinecheck=off,width=16.5cm}
\begin{longtable}[c]{|P{2cm}|P{2cm}|P{5cm}|P{6cm}|}
    \caption{Комплекс работ\label{tbl:works}}
    \\ \hline
        \textbf{Номер события} & 
        \textbf{Цифры работ} & 
        \textbf{Продолжительность работ (недель)} & 
        \textbf{Наименование и содержание работ} 
    \\ \hline
        1 &
        --- &
        0 &
        ---    
    \\ \hline
        2 &
        1-2 &
        3 &
        Разработка бренда
    \\ \hline
        3 &
        2-3 &
        3 &
        Разработка дизайна чехлов
    \\ \hline
        4 &
        1-4 &
        3 &
        Разработка рекламных средств
    \\ \hline
        4 &
        3-4 &
        2 &
        Внедрение дизайна в рекламу
    \\ \hline
        5 &
        4-5 &
        6 &
        Распространение рекламы
    \\ \hline
        6 &
        1-6 &
        4 &
        Заключение договора на поставку материалов
    \\ \hline
        6 &
        3-6 &
        2 &
        Согласование дизайна и материалов
    \\ \hline
        7 &
        6-7 &
        3 &
        Подготовка форм для разных моделей
    \\ \hline
        8 &
        7-8 &
        4 &
        Литье форм
    \\ \hline
        9 &
        8-9 &
        2 &
        Нанесение изображений
    \\ \hline
        10 &
        5-10 &
        8 &
        Сбор заказов
    \\ \hline
        10 &
        9-10 &
        1 &
        Тестирование
    \\ \hline
        11 &
        10-11 &
        3 &
        Передача заказчику
    \\ \hline
\end{longtable}
}
%\end{center}


\chapter{Построение сетевого графика}

\section{Определение ранних начал работ}

При определении ранних начал работ расчет ведется слева направо от исходного
события к завершающему. За величину раннего начала принимается \bfit{наибольшая
продолжительность из всех путей, ведущих к данному событию}.

Определение ранних начал работ представлены в таблице \ref{tbl:work-begin}.

\vspace{-0.5cm}
\begin{center}
\captionsetup{format=hang,justification=raggedright,
              singlelinecheck=off,width=12cm}
    \begin{longtable}[c]{|P{2cm}|P{2cm}|P{7cm}|}
    \caption{Ранние начала работ\label{tbl:work-begin}}
    \\ \hline
        \textbf{Событие}
        & \textbf{Цифры работ}
        & \textbf{Недель}
    \\ \hline
        1 & --- & 0
    \\ \hline
        2 & 1--2 & 0 + 3 = 3
    \\ \hline
        3 & 2--3 & 3 + 3 = 6
    \\ \hline
        \multirow{2}{*}{4} & 1--4 & 0 + 3 = 3\\
         & 3--4 & 6 + 2 = 8 ($\max$)
    \\ \hline
        5 & 4--5 & 8 + 6 = 14
    \\ \hline
        \multirow{2}{*}{6} & 1--6 & 0 + 4 = 4\\
         & 4--6 & 6 + 2 = 8 ($\max$)
    \\ \hline
        7 & 6--7 & 8 + 3 = 11
    \\ \hline
        8 & 7--8 & 11 + 4 = 15
    \\ \hline
        9 & 8--9 & 15 + 2 = 17
    \\ \hline
        \multirow{2}{*}{10} & 5--10 & 14 + 8 = 22 ($\max$)\\
         & 9--10 & 17 + 1 = 18
    \\ \hline
        11 & 10--11 & 22 + 3 = 25
    \\ \hline
\end{longtable}
\end{center}

\clearpage
\section{Определение поздних начал работ}

Расчет поздних начал работ ведется справа налево от завершающего к начальному
событию графика. За величину позднего начала принимается \bfit{наименьшая
продолжительность из всех путей, ведущих из данного события}.

Определение поздних начал работ представлены в таблице \ref{tbl:work-end}.

\vspace{-0.5cm}
\begin{center}
\captionsetup{format=hang,justification=raggedright,
              singlelinecheck=off,width=12cm}
    \begin{longtable}[c]{|P{2cm}|P{2cm}|P{7cm}|}
    \caption{Поздние начала работ\label{tbl:work-end}}
    \\ \hline
        \textbf{Событие}
        & \textbf{Цифры работ}
        & \textbf{Недель}
    \\ \hline
        11 & --- & 25
    \\ \hline
        10 & 11--10 & 25 - 3 = 22
    \\ \hline
        9 & 10--9 & 22 - 1 = 21
    \\ \hline
        8 & 9--8 & 21 - 2 = 19
    \\ \hline
        7 & 8--7 & 19 - 4 = 15
    \\ \hline
        6 & 7--6 & 15 - 3 = 12
    \\ \hline
        5 & 10--5 & 22 - 8 = 14
    \\ \hline
        4 & 5--4 & 14 - 6 = 8
    \\ \hline
        \multirow{2}{*}{3} & 6--3 & 12 - 2 = 10\\
         & 4--3 & 8 - 2 = 6 ($\min$)
    \\ \hline
        2 & 3--2 & 6 - 3 = 3
    \\ \hline
        \multirow{3}{*}{1} & 6--1 & 12 - 4 = 8\\
         & 4--1 & 4 - 3 = 1\\
         & 2--1 & 3 - 3 = 0 ($\min$)
    \\ \hline
\end{longtable}
\end{center}

\clearpage
\section{Сетевой график и критический путь}

На рисунке~\ref{img:scheme} преставлен сетевой график выполнения работ.

\imgs{scheme}{h!}{1.1}{Сетевой график выполнения комплекса работ}

\textbf{Критический путь} продолжительностью 25 недель определяется
последовательностью событий:

\begin{center}
    1 --- 2 --- 3 --- 4 --- 5 --- 10 --- 11
\end{center}

\chapter{Прибыль предприятия}

\section{Номенклатура переменных и постоянных затрат}

Номенклатура переменных затрат представлена в таблице \ref{tbl:var}, а
постоянных --- в таблице \ref{tbl:const}.

\vspace{-0.5cm}
\begin{center}
\captionsetup{format=hang,justification=raggedright,
              singlelinecheck=off,width=13.5cm}
    \begin{longtable}[c]{|P{6cm}|P{7cm}|}
    \caption{Переменные затраты\label{tbl:var}}
    \\ \hline
        \textbf{Наименование} & \textbf{Рублей на единицу продукции}
    \\ \hline
        Силикон & $180$
    \\ \hline
        Краска & $120$
    \\ \hline
        Электричество & $50$
    \\ \hline
        Упаковка & $170$
    \\ \hline
        Транспортировка & $200$
    \\ \hline
        \textbf{Итого} & $\textbf{720}$
    \\ \hline
\end{longtable}
\end{center}


\vspace{-2cm}
\begin{center}
\captionsetup{format=hang,justification=raggedright,
              singlelinecheck=off,width=13.5cm}
    \begin{longtable}[c]{|P{6cm}|P{7cm}|}
    \caption{Постоянные затраты\label{tbl:const}}
    \\ \hline
        \textbf{Наименование} & \textbf{Млрд руб в год}
    \\ \hline
        Заработная плата & $0.1$
    \\ \hline
        Аренда помещений & $0.7$
    \\ \hline
        Коммунальные расходы & $0.4$
    \\ \hline
        Реклама & $0.05$
    \\ \hline
        Амортизация & $0.3$
    \\ \hline
        \textbf{Итого} & $\textbf{1.55}$
    \\ \hline
\end{longtable}
\end{center}

\clearpage
\section{Рассчет прибыли предприятия}

Средняя цена реализации продукта равно $900$ руб, а планируемый объем
производства в год --- $16\ 000\ 000$ штук. Тогда:
\begin{equation}
    \textbf{Выручка от реализации} = 900 \cdot 16\ 000\ 000 = 14.4 \text{ млрд
    руб.}
\end{equation}

Переменные затраты на единицу продукции составляют $720$ руб, тогда для
всего объема производства в год:
\begin{equation}
    \textbf{Переменные затраты} = 720 \cdot 16\ 000\ 000 = 11.52 \text{ млрд руб.}
\end{equation}

Маржинальный доход, вычисляющийся как разность выручки от
реализации и переменных затрат, составляет
\begin{equation}
    \textbf{Маржинальный доход} = 14.4 - 11.52 = 2.88 \text{ млрд руб.}
\end{equation}

Постоянные расходы составляют $1.55$ млрд руб в год. Годовая прибыль
вычисляется как разность маржинального дохода и постоянных расходов и составляет
в данном случае:
\begin{equation}
    \textbf{Годовая прибыль} = 2.88 - 1.55 = 1.33 \text{ млрд руб.}
\end{equation}

Cреднюю величину маржинального дохода (СВМД для краткости записи) на единицу
продукции:
\begin{equation}
    \textbf{СВМД} = \frac{2\ 880\ 000\ 000}{16\ 000\ 000} = 180 \text{ руб.}
\end{equation}

В точке безубыточности прибыль равна нулю, поэтому эта точка может быть найдена
при условии равенства выручки и суммы переменных и постоянных затрат, то есть
получается следующее уравнение, где $X$ --- точка безубыточности:
\begin{equation}
    \begin{gathered}
        900 \cdot X = 720 \cdot X + 1\ 550\ 000\ 000
        \\ 180 \cdot X = 1\ 550\ 000\ 000
        \\ X = \frac{1\ 550\ 000\ 000}{180} = 8.6 \text{ млн штук}
    \end{gathered}
\end{equation}

\clearpage
\section{Результаты вычислений}

Результаты вычислений представлены в обощающей таблице \ref{tbl:result} прибыли
предприятия при установленном объеме реализации продукции.
\begin{center}
    \captionsetup{justification=raggedright,singlelinecheck=off}
    \begin{longtable}[c]{|P{2cm}|l|P{2cm}|}
    \caption{Прибыль предприятия\label{tbl:result}}
    \\ \hline
        \textbf{№ п/п} & \textbf{Показатели} & \textbf{Значение}
    \\ \hline
        1 & Объем производства, млн штук & $16$
    \\ \hline
        2 & Выручка от реализации, млрд руб & $14.4$
    \\ \hline
        3 & Перенные затраты, млрд руб & $11.52$
    \\ \hline
        4 & Маржинальный доход, млрд руб & $2.88$
    \\ \hline
        5 & Постоянные затраты, млрд руб & $1.55$
    \\ \hline
        6 & Прибыль, млрд руб & $1.33$
    \\ \hline
        7 & Средняя величина маржинального дохода, руб & $180$
    \\ \hline
        8 & Точка безубыточности, млн штук& $8.6$
    \\ \hline
\end{longtable}
\end{center}
