\documentclass{bmstu}

% table of contents
\usepackage{tocloft}
\renewcommand{\cfttoctitlefont}{\hfill\large\bfseries}
\renewcommand{\cftaftertoctitle}{\hfill\hfill}
\renewcommand{\cftchapleader}{\cftdotfill{\cftdotsep}}
\setlength\cftbeforetoctitleskip{-22pt}
\setlength\cftaftertoctitleskip{15pt}
\setcounter{secnumdepth}{4}

% searchable and copyable
\usepackage{cmap}

% Times New Roman
\usepackage{pscyr}
\renewcommand{\rmdefault}{ftm}

% Links
\def\UrlBreaks{\do\/\do-\do\_}
\usepackage[nottoc]{tocbibind} % for bib link
\usepackage[numbers]{natbib}
\renewcommand*{\bibnumfmt}[1]{#1.}
%% Captions
\usepackage[figurename=Рисунок,labelsep=endash,
            justification=centering]{caption}
\usepackage{subcaption}
\usepackage{hyphenat}


%% Images
\usepackage{graphicx}
\usepackage{wrapfig}

\newcommand{\img}[4] {
	\begin{figure}[ht!]
		\center{
            \includegraphics[height=#1]{../data/img/#2}
            \caption{#3}
            \label{img:#4}
        }
	\end{figure}
}

\newcommand{\twoimg}[7] {
    \begin{figure}[ht!]
    \centering
        \begin{minipage}[h]{0.4\linewidth}
            \begin{center}
                \includegraphics[height=#1]{../data/img/#2}
                \caption{#3}
                \label{img:#4}
            \end{center}
        \end{minipage}
        \hspace{8ex}
        \begin{minipage}[h]{0.4\linewidth}
            \begin{center}
                \includegraphics[height=#1]{../data/img/#5}
                \caption{#6}
                \label{img:#7}
            \end{center}
        \end{minipage}
    \end{figure}
}

% Lists
\usepackage{enumitem}
\renewcommand{\labelitemi}{$-$}
\setlist{nosep, leftmargin=\parindent, wide}
%
%% Comments
%\usepackage{comment}
%
% Text
\usepackage{amstext}
\usepackage{ulem}
\renewcommand{\ULdepth}{1.5pt}

% Formulas
\DeclareMathOperator*{\argmax}{argmax}

% Tabulars
\usepackage{threeparttable}
\usepackage{csvsimple}
\usepackage{longtable,ltcaption,booktabs}
\usepackage{tabularx}
\usepackage{multirow}

% PDF
\usepackage{pdfpages}

% cites
\usepackage{natbib}
\bibpunct{[}{]}{;}{n}{}{}

% Listings
\usepackage{listings}
\usepackage[newfloat]{minted}
\usepackage{verbatim}
\usepackage[framemethod=tikz]{mdframed}

\mdfdefinestyle{mymdstyle}{
    innerleftmargin=0mm,
    innerrightmargin=0mm,
    innertopmargin=-4pt,
    innerbottommargin=-9pt,
    splittopskip=\baselineskip,
    hidealllines=true,
    middleextra={
      \node[anchor=west] at (O|-P)
        {\lstlistingname~\thelstlisting~(продолжение)};
        },
    secondextra={
      \node[anchor=west] at (O|-P)
        {\lstlistingname~\thelstlisting~(продолжение)};},
}

\surroundwithmdframed[style=mymdstyle]{lstlisting}
\newmdenv[style=mymdstyle]{mdlisting}

\newcommand{\mylisting}[4] {
    \noindent
    \begin{minipage}{\linewidth}
        \captionsetup{justification=raggedright,singlelinecheck=off}
        \begin{lstinputlisting}[
            caption={#1},
            label={lst:#2},
            linerange={#3}
        ]{#4}
        \end{lstinputlisting}
    \end{minipage}
}

\newcommand{\mybreaklisting}[4]{
    \begin{mdlisting}
        \captionsetup{justification=raggedright,singlelinecheck=off}
        \lstinputlisting[label=lst:#2, caption=#1,
        linerange=#3]{}
    \end{mdlisting}
}


\include {listings.inc}

\begin{document}

\chapter{Абиотический фактор вода}

В жизни всех организмов вода выступает как важнейший экологический фактор.

\section{Процессы в биосфере, происходящие при участии воды}

Без воды на нашей планете не могло бы быть жизни. Вода важна для живых
организмов вдвойне, так как она не только необходимый компонент живых клеток, но
для многих организмов еще и среда обитания.

Из всех жидких и твердых веществ у воды \textbf{наибольшая теплоемкость}.
Поэтому прогревшись в течение лета, моря и океаны медленно остывают зимой,
отдавая тепло атмосфере. С этим связано значительное \textbf{постоянство
температурных условий} водной среды обитания.

Первостепенное значение во всех проявлениях жизнедеятельности имеет
\textbf{водный обмен между организмами и внешней средой}.

Степень насыщения воздуха и почвы водяными парами --- \textbf{влажность} ---
имеет большое значение для обитателей суши и нередко является фактором,
\textbf{лимитирующим распространение и численность организмов Земли}.

(Например, степные и особенно лесные растения требуют повышенного содержания
паров в воздухе, растения же пустынь приспособились к низкой влажности.)

Влажность и осадки являются факторами, влияющими на \textbf{формирование
климатических условий}.

При практически одинаковых географических условиях на Земле существуют и жаркая
пустыня, и тропический лес. Различие состоит только в годовом количестве
осадков: в первом случае 0,2-200~мм, а во втором 900-2000~мм.

\textbf{Осадки}, тесно связанные с влажностью воздуха, представляют собой
результат конденсации и кристаллизации водяных паров в высоких слоях атмосферы.
В приземном слое воздуха при высокой относительной влажности и необходимого
уровня температуры образуются росы, туманы, а при низких температурах
наблюдается кристаллизация влаги --- выпадает иней.

\textbf{Малое количество осадков}, быстрый дренаж, интенсивное испарение либо
сочетания этих факторов ведут к \textbf{иссушению}, а \textbf{избыток влаги} ---
к \textbf{переувлажнению и заболачиванию} почв.

Помимо отмеченного, влажность воздуха как экологический фактор при своих крайних
значениях (повышенной и пониженной влажности), \textbf{усиливает воздействие
температуры на организм}. Это описывает такой параметр, как дефицит влажности.

\textbf{Дефицит влажности} --- разность между максимально возможным и фактически
существующим насыщением при данной температуре. Это один из важнейших
экологических параметров, поскольку характеризует сразу две величины:
температуру и влажность. Чем выше дефицит влажности, тем суше и теплее, и
наоборот.

Режим осадков (распределение осадков по месяцам) --- важнейший фактор,
определяющий миграцию загрязняющих веществ в природной среде и вымывание их из
атмосферы.

\section{Роль воды в биологических процессах живых организмов}

Живых организмов, не содержащих воду, на Земле не найдено. В период
\textbf{активной жизнедеятельности} растений и животных содержание воды в их
организмах, как правило, \textbf{80...90\% от массы тела}. В \textbf{состоянии
покоя} количество воды в организме может значительно \textbf{снижаться, однако
она не исчезает полностью}. Например, в сухих мхах и лишайниках содержание воды
в общей массе организмов составляет 5...7\%.

Наземные организмы вынуждены \textbf{постоянно пополнять запасы воды}. Поэтому у них в
процессе эволюции выработались приспособления, регулирующие водный обмен и
обеспечивающие экономное расходование влаги. Эти приспособления могут выражаться
в изменении структур (плотные внешние оболочки), модификации физиологических
процессов (более экономное использование воды в организме).

Одни животные пустыни получают воду из пищи, другие за счет окисления
своевременно запасенных жиров (например, верблюд, способный путем биологического
окисления из 100~г жира получить 107~г метаболической воды); при этом у них
минимальна водопроницаемость наружных покровов тела, преимущественно ночной
образ жизни и т.~д.

При периодической засушливости для живых организмов характерно впадание в
\textbf{состояние покоя с минимальной интенсивностью обмена веществ}. 

Все способы адаптации организмов для поддержанием количества воды в
теле связаны с большим числом биологических процессов, которые обеспечивают
возможность существования.

В биологических процессах живых организмов вода выполняет следующие функции:

\begin{itemize}
    \item \textbf{основная часть протоплазмы клеток, тканей, растительны и
        животных соков};
    \item все \textbf{биохимические процессы в организме} (синтез и распад органического
        вещества, газообмен) осуществляются при достаточном обеспечении водой;
    \item Вода с растворенными в ней веществами обусловливает осмотическое
        давление клеточных и тканевых жидкостей, обеспечивает
        \textbf{межклеточный обмен};
    \item \textbf{наибольшая теплоемкость} $\Rightarrow$ биохимические процессы в
        клетках живых организмов протекают в \textbf{стабильных условиях}, что
        обеспечивает их высокую эффективность;
    \item \textbf{поверхностное натяжение и плотность} воды определяют
        \textbf{высоту}, на которую она может подниматься в капиллярных системах
        проводящих тканей у растений (вода обладает наибольшим поверхностным
        натяжением из всех известных жидкостей, за исключением ртути, что имеет
        огромное значение для жизни растительного мира);
    \item вода является превосходным растворителем для многих веществ. Когда
        вещество переходит в раствор, его молекулы и ионы получают возможность
        двигаться более свободно и, соответственно, его реакционная способность
        возрастает. По этой причине в клетке \textbf{большая часть химических
            реакций протекает в водных растворах};
    \item присущие воде свойства растворителя означают также, что вода служит
        \textbf{средой для транспорта} различных веществ (кровеносная система у
        животных, проводящие системы растений);
    \item \textbf{большая теплота испарения воды}, обусловленная водородными
        связями в молекулах воды, также играет очень важную роль в
        жизнедеятельности организмов. Испарение сопровождается
        \textbf{охлаждением поверхности тела}. Это используется при
        потоотделении (у животных), при транспирационном охлаждении листьев (у
        растений);
    \item вода служит \textbf{средой для оплодотворения}.
\end{itemize}

Таким образом вода:

\begin{itemize}
    \item у всех организмов:
        \begin{itemize}
            \item обеспечивает поддержание структуры (высокое содержание воды в
                протоплазме);
            \item служит растворителем и средой для диффузии;
            \item участвует в реакциях гидролиза;
            \item служит средой, в которой проходит оплодотворение.
        \end{itemize}
    \item у растений:
        \begin{itemize}
            \item обеспечивает транспорт неорганических ионов и органических
                молекул;
            \item прорастание семян (набухание разрыв семенной кожуры и
                дальнейшее развитие);
            \item участвует в фотосинтезе (на молекулярном уровне) и
                транспирации, т.~е. испаряется с поверхности листьев, охлаждая
                их,
        \end{itemize}
    \item у животных: 
        \begin{itemize}
            \item обеспечивает транспорт веществ внутри организма;
            \item способствует охлаждению тела (потоотделение);
            \item служит одним из компонентов смазки (например в суставах);
            \item обеспечивает опорные функции (гидростатический скелет);
            \item выполняет защитную функцию.
        \end{itemize}
\end{itemize}

\end{document}
