\chapter{Конфликты: понятие, сущность, классификация}

\section{Понятие конфликта}

Существуют различные определения конфликта, однако все они так или иначе
отражают значение производящего латинского слова \textit{conflictus} ---
столкновение~\cite{book06}. Под конфликтом можно понимать как собственно
столкновение противоположных позиций, так и способ или действия разрешения
значимых противоречий, возникших в результате взаимодействия~\cite{book04},
поэтому в наиболее общем виде конфликту можно дать следующее определение.

\textbf{Конфликт} --- это такое отношение между субъектами социального
взаимодействия, которое характеризуется их противоборством на основе
противоположно направленных мотивов (потребностей, интересов, целей, идеалов,
убеждений) или суждений (мнений, взглядов, оценок и т.~п.)~\cite{book03}.

Первоначально в определение конфликта также включали негативные чувства и
враждебные действия его учатников, однако современные подходы основаны на том,
что конфликт может протекать и без агрессии, в рамках конструктивного
взаимодействия.

\section{Сущность конфликта}

Cущностью любого конфликта ялвяется противоборство или противодействие, а не
просто существующее противоречие или отсутствие согласия~\cite{book07}, что говорит о
динамических свойствах конфликта. Таким образом, любой конфликт можно
рассматривать с двух сторон: статической и динамической.

Со статической стороны рассмативают структуру конфликта --- совокупность
устойчивых связей конфликта, обеспечивающих его целостность, тождественность
самому себе, отличие от других явлений социальной жизни, без которых он
не может существовать как динамически взаимосвязанная
система~\cite{book04}.

Структура конфликта представлена на рисунке~\ref{img:01}, на котором
приняты следующие обозначения~\cite{book03}:
\begin{itemize}
    \item \textbf{П} --- \textit{предмет конфликта} --- то, из-за чего возникает
        конфликт;
    \item \textbf{С1, С2} --- \textit{стороны или субъекты конфлита} ---
        это субъекты социального взаимодействия, находящиеся в
        состоянии конфликта или же явно или неявно поддерживающие
        конфликтующих;
    \item \textbf{М1, М2} --- \textit{мотивы конфлита} --- это внутренние
        побудительные силы, подталкивающие субъектов социального взаимодействия
        к конфликту, в форме которых выступают потребности, интересы,
        цели, идеалы, убеждения;
    \item \textbf{ОК1, ОК2} --- \textit{образы конфликта} --- это
        отображение предмета конфликта в сознании субъектов конфликтного
        взаимодействия;
    \item \textbf{ПС1, ПС2} --- \textit{позиции сторон} --- это то,
        о чем они заявляют друг другу в ходе конфликта или в переговорном
        процессе.
\end{itemize}

\img{7cm}{conflicts}{Структура конфликта}{01}

С динамической стороны конфликт рассматривают как процесс,
разворачивающийся во времени, включающий определнные стадии и переходы от
одной стадии к другой.

Выделяют следующие этапы протекания конфликта:
\begin{enumerate}
    \item
\end{enumerate}

Каждый из этапов относится к одному из периодов.
\begin{enumerate}
    \item
\end{enumerate}

Соответсвие между этапами и периодами приведено в таблице~\ref{tab:01}.

Также выделяют фазы, которые непосредственно связаны с этапами конфликта и
отражают динамику конфликта с точки зрения реальных возможностей его
разрешения. Основными фазами конфликта являются:
\begin{enumerate}
    \item
\end{enumerate}

Важно помнить, что фазы конфликта могут повторятся циклически, что отражено
на рисунке~\ref{img:02}.

Бла-бла-бла, вывод

\section{Основные способы классификации}
Основные способы классификации + расписать немного те, о которых далее не будет
сказано

