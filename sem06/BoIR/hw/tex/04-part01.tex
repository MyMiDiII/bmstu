\chapter{Конфликты: понятие, сущность, классификация}

\section{Понятие конфликта}

Существуют различные определения конфликта, однако все они так или иначе
отражают значение производящего латинского слова \textit{conflictus} ---
столкновение~\cite{book06}. Под конфликтом можно понимать как собственно
столкновение противоположных позиций, так и способ или действия разрешения
значимых противоречий, возникших в результате взаимодействия~\cite{book04},
поэтому в наиболее общем виде конфликту можно дать следующее определение.

\textbf{Конфликт} --- это такое отношение между субъектами социального
взаимодействия, которое характеризуется их противоборством на основе
противоположно направленных мотивов (потребностей, интересов, целей, идеалов,
убеждений) или суждений (мнений, взглядов, оценок и т.~п.)~\cite{book03}.

Первоначально в определение конфликта также включали негативные чувства и
враждебные действия его участников, однако современные подходы основаны на том,
что конфликт может протекать и без агрессии, в рамках конструктивного
взаимодействия.

\section{Сущность конфликта}

Сущностью любого конфликта является противоборство или противодействие, а не
просто существующее противоречие или отсутствие согласия~\cite{book07}, что говорит о
динамических свойствах конфликта. Таким образом, любой конфликт можно
рассматривать с двух сторон: статической и динамической.

Со статической стороны рассматривают структуру конфликта --- совокупность
устойчивых связей конфликта, обеспечивающих его целостность, тождественность
самому себе, отличие от других явлений социальной жизни, без которых он
не может существовать как динамически взаимосвязанная
система~\cite{book04}.

Структура конфликта представлена на рисунке~\ref{img:01}, на котором
приняты следующие обозначения~\cite{book03}:
\begin{itemize}
    \item \textbf{П} --- \textit{предмет конфликта} --- то, из-за чего возникает
        конфликт;
    \item \textbf{С1, С2} --- \textit{стороны или субъекты конфликта} ---
        это субъекты социального взаимодействия, находящиеся в
        состоянии конфликта или же явно или неявно поддерживающие
        конфликтующих;
    \item \textbf{М1, М2} --- \textit{мотивы конфликта} --- это внутренние
        побудительные силы, подталкивающие субъектов социального взаимодействия
        к конфликту, в форме которых выступают потребности, интересы,
        цели, идеалы, убеждения;
    \item \textbf{ОК1, ОК2} --- \textit{образы конфликта} --- это
        отображение предмета конфликта в сознании субъектов конфликтного
        взаимодействия;
    \item \textbf{ПС1, ПС2} --- \textit{позиции сторон} --- это то,
        о чем они заявляют друг другу в ходе конфликта или в переговорном
        процессе.
\end{itemize}

\img{7cm}{conflicts}{Структура конфликта}{01}

С динамической стороны конфликт рассматривают как процесс,
разворачивающийся во времени, включающий определенные стадии и переходы от
одной стадии к другой.

Выделяют следующие этапы протекания конфликта~\cite{art01}:

\begin{enumerate}[label=\arabic*)]
    \item \textbf{объективное противоречие} --- объективные внешние
        обстоятельства и условия, способствующие возникновению
        конфликта~\cite{book01};
    \item осознание объективного противоречия;
    \item попытка разрешить противоречие неконфликтными способами;
    \item \textbf{конфликтная ситуация} --- это ситуация скрытого или открытого
        противоборства двух или нескольких участников, каждый из которых имеет
        свои цели, мотивы, средства и способы решения проблемы, имеющей
        личную значимость для каждого из ее участников~\cite{book04};
    \item \textbf{инцидент конфликта} --- это действие или совокупность действий
        участников конфликтной ситуации, провоцирующее резкое обострение
        противоречия и начало борьбы между ними~\cite{book04};
    \item \textbf{эскалация} --- резкая интенсификация борьбы оппонентов~\cite{book04};
    \item \textbf{сбалансированное противодействие} --- конфликтующие
        стороны продолжают противодействовать, но
        интенсивность борьбы снижается, однако действия по достижению
        согласия не предпринимаются~\cite{book04};
    \item завершение конфликта;
    \item частичная нормализация отношений;
    \item полная нормализация отношений.
\end{enumerate}

Каждый из этапов относится к одному из периодов~\cite{book01}.
\begin{enumerate}[label=\arabic*)]
    \item предконфликтный;
    \item открытый (собственно конфликт);
    \item послеконфликтный.
\end{enumerate}

Соответствие между этапами и периодами приведено в таблице~\ref{tab:01}.

\noindent
{
\captionsetup{format=hang,justification=raggedright,
              singlelinecheck=off,width=11cm}
\begin{longtable}[c]{|p{4.2cm}|p{6cm}|}
\caption{Основные этапы и периоды конфликта\label{tab:01}}\\
    \hline
    \bfseries Периоды & \bfseries Этапы \\
    \hline
    \multirow{4}{*}{\parbox[c]{4cm}{\vspace{1.5cm}\centeringПредконфликтный период}}
          & объективное противоречие \\ \cline{2-2}
          & осознание объективного противоречия \\ \cline{2-2}
          & попытка неконфликтного разрешения противоречия \\ \cline{2-2}
          & конфликтная ситуация \\
    \hline
    \multirow{4}{*}{\parbox[c]{4cm}{\vspace{1cm}\centeringОткрытый период}}
          & инцидент конфликта \\ \cline{2-2}
          & эскалация \\ \cline{2-2}
          & сбалансированное\newlineпротиводействие \\ \cline{2-2}
          & завершение конфликта \\
    \hline
    \multirow{2}{*}{\parbox[c]{4cm}{\vspace{0.9cm}\centeringПослеконфликтный период}}
          & частичная нормализация\newlineотношений \\ \cline{2-2}
          & полная нормализация\newlineотношений \\ \cline{1-2}
\end{longtable}
}

Также выделяют фазы, которые непосредственно связаны с этапами конфликта и
отражают динамику конфликта с точки зрения реальных возможностей его
разрешения. Основными фазами конфликта являются:
\begin{enumerate}
    \item начальная фаза;
    \item фаза подъема;
    \item пик конфликта;
    \item фаза спада.
\end{enumerate}

Важно помнить, что фазы конфликта могут повторятся циклически, что отражено
на рисунке~\ref{img:02}.
~\\

\img{5cm}{phase}{Фазы конфликта}{02}

Важно иметь в виду, что все перечисленные выше периоды и этапы конфликта могут
иметь различную длительность. Некоторые этапы могут опускаться или занимать
настолько незначительный промежуток времени, что практически отсутствует
возможность различить их~\cite{book01}.

\section{Основные способы классификации}

Конфликты, представляющие собой сложное социально-психологическое явление,
весьма многообразны и их можно классифицировать по различным признакам. С
практической точки зрения классификация конфликтов важна, так как она позволяет
ориентироваться в их специфических проявлениях и, следовательно, помогает
оценить возможные пути их разрешения~\cite{book03}.

Обобщая наиболее распространенные классификации конфликтов, можно выделить такие
базовые основания классификации и типологии конфликтов:

\begin{itemize}
    \item \textbf{по составу и количеству конфликтующих сторон или участников
        конфликта, или по объему} (внутриличностные, межличностные,
        личностно-групповые, межгрупповые, межколлективные, межгосударственные,
        межпартийные, межнациональные);
    \item \textbf{длительности} (кратковременные и затяжные);
    \item \textbf{источнику возникновения} (объективно и субъективно
        обусловленные)~\cite{book08};
    \item \textbf{проблемно-деятельностному признаку} (управленческие, семейные,
        педагогические, политические, экономические, творческие);
    \item \textbf{времени протекания конфликта} (острые и хронические;
        быстротекущие, длительные, вялотекущие и др.);
    \item \textbf{содержанию конфликта} (содержательные, или проблемные, и
        «коммунальные»);
    \item \textbf{тенденции к преобразованиям и возможностям разрешения
        конфликта} (конструктивные и деструктивные, или неконструктивные);
    \item \textbf{степени остроты противоречий} (недовольство, разногласие,
        противодействие, раздор, вражда, война и др.);
    \item \textbf{степени интенсивности конфликта} (основные и неосновные,
        реалистические и нереалистические и т.~д.)~\cite{book02}.
\end{itemize}

Некоторые типологии подробнее описаны в разделе~\ref{head:03}.
