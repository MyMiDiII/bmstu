\chapter*{ВВЕДЕНИЕ}
\addcontentsline{toc}{chapter}{ВВЕДЕНИЕ}

Жизнь человека в обществе сложна и полна противоречий, которые часто приводят к
столкновению интересов как отдельных людей, так и больших и малых социальных
групп. При этом конфликты, переживаемые человеком, являются важнейшим источником
его развития, определяют конструктивный или деструктивный жизненный путь. То
есть конфликт может привести к положительным изменениям как в жизни одного
человека, так и в жизни общества, а правильное поведение в конфликте может
направить его в такое русло, чтобы обеспечить минимизацию неизбежных социальных
потерь и устранить негативные последствия для интересов личности, общества и
государства. Но для правильного поведения в конфликте и даже управления им
необходимы хотя бы базовые знания о конфликтах, попытка описания которых
делается в данной работе \cite{book01,book02}.

Таким образом, целью данной работы является описание базовых теоретических
сведений о понятии конфликта. Для достижения цели выполняются следующие задачи:
\begin{itemize}
    \item дается определение понятия конфликт;
    \item описывается его сущность, причины и механизм возникновения;
    \item приводятся способы классификации конфликтов и их соответствующие типы.
\end{itemize}
