\chapter{Причины и механизм возникновения конфликтов}

\section{Причины}

\textbf{Причины конфликта} — это явления, события, факты, ситуации, которые
предшествуют конфликту и, при определенных условиях деятельности субъектов
социального взаимодействия, вызывают их~\cite{book04}.

Существует несколько подходов к классификации причин конфликтов. В
качестве непосредственных причин конфликта можно рассматривать различные
факторы:

\begin{itemize}
    \item информационные (неприемлемость информации, дезинформация,
        несвоевременность информации~и~т.~д.);
    \item поведенческие (неуместность, грубость, бестактность,
        экоистичность~и~т.~д.);
    \item факторы отношений (неудовлетворенность от
        взаимодействия между сторонами);
    \item ценностные (принципы, предрассудки, предпочтения,
        приоритеты, нравственные ценности~и~т.~д.);
    \item структурные (относительно стабильные объективные
        обстоятельства, трудо поддающиеся изменению, например, власть, право
        собствености~и~т.~д.).
\end{itemize}

Кроме того, различают объективные и субъективные причины конфликтности.

Объективными причинами считаются обстоятельства социального взаимодействия,
приводящие к конфликту (например, отсутствие объективной системы оценки
персонала, Несоответствие прав и обязанностей работников, низкий уровень
заработной платы, недостаток ресурсов~и~т.~п.). Эти причины приводят к созданию
предконфликтной обстановки --- объективного компонента конфликтной ситуации.

Субъективные причины обусловлены
индивидуально-психологическими особенностями и непосредственным взаимодействием
людей (например, неверные дейстия руководителя или подчиненных, манипулирование,
психологическая несовместимость работников~и~т.~п.). Конфликты
субъективного характера нередко вызываются психологической несовместимостью
людей, обусловленной особенностями их характеров и темпераментов.

Причины конфликтов, помимо подразделения на группы объективных и субъективных,
могут быть представлены в виде общих и частных. К общим причинам относят:
\begin{itemize}
    \item социально-политические (связанные с социально-политической и
        экономической ситуацией в стране);
    \item социально-демографические (отражают различия в установках и мотивах
        людей, обусловленные их полом, возрастом, принадлежностью к этническим
        группам и др.);
    \item социально-психологические (отражают социально-психологические явления
        в социальных группах: взаимоотношения, лидерство, групповые мотивы,
        коллективные мнения, настроения и т.д.);
    \item индивидуально-психологические (отражают индивидуальные
        психологические особенности личности: темперамент, характер,
        способности, мотивы и т.~п.).
\end{itemize}

Частные причины непосредственно связаны с конкретным видом конфликта,
например:
\begin{itemize}
    \item неудовлетворенность условиями деятельности;
    \item нарушение служебной этики;
    \item нарушение трудового законодательства;
    \item ограниченность ресурсов;
    \item различие в целях, ценностях, средствах достижения целей;
    \item неудовлетворенность коммуникации.
\end{itemize}

\section{Механизм}

Как было рассмтрено выше, конфликты позникают по различным причинам,
причем большинство из них, как считают специалисты, возникают помимо желания
их участников. Поэтому для того, чтобы недопустить конфликт или успешно
разрешить его, необходимо механизм возникновения конфликтов~\cite{art02}.

Конфликтологом В.~П.~Шейновым было выделено три формулы
конфликтов:

\begin{itemize}
    \item На основе конфликтогенов: слов, действий (или бездействий),
        поведенческих актов или поведения в целом, --- которые могут привести
        к конфликту. Формула в таком случае имеет следующий вид:
        
        $$ \text{КФ}_1 + \text{КФ}_2 + ... + \text{КФ}_n = \text{К}, $$

      где $\text{КФ}_1, \text{КФ}_2, ..., \text{КФ}_n$ --- конфликтогены,
      
      ~~~~~~$\text{К}$ --- конфликт.

      При этом важно, что каждый последующий конфликтоген является более
      сильным, чем предыдущий (так называемый, закон эскалации
      конфликтогенов):

      $$\text{КФ}_1 < \text{КФ}_2 < ... < \text{КФ}_n.$$

      Решение данного вида конфликта заключается втом, чтобы не
      использовать слова или действия, которые могут привести к конфликту.
    \item На основе конфликтной ситуации и инцидента:

      $$\text{КС} + \text{И} = \text{К},$$

     где $\text{КС}$ --- конфликтная ситуация,

       ~~~~~~$\text{И}$ --- инцидент.

       ~~~~~~$\text{К}$ --- конфликт.

    Для того, чтобы разрешить конфликт при указанной формуле, необходимо
    его обговорить и разрешить призошедший инцидент.
    
    \item На основе нескольких конфликтных ситуациях:

        $$ \text{КС}_1 + \text{КС}_2 + ... + \text{КС}_n = \text{К}, $$

      где $\text{КС}_1, \text{КС}_2, ..., \text{КС}_n$ --- конфликтогены,
      
      ~~~~~~$\text{К}$ --- конфликт.

      Для того, чтобы разрешить конфликт, возникающий по данной формуле,
      необходимо уладить все имеющиеся конфликтные ситуации.
\end{itemize}


