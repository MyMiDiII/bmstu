\chapter{\label{head:03}Классификация конфликтов}

\section{По объему}

По объему конфликты подразделяют на~\cite{book08}:
\begin{itemize}
    \item внутриличностные;
    \item межличностные;
    \item между личностью и группой (внутригрупповые);
    \item межгрупповые.
\end{itemize}

При \textbf{внутриличностном конфликте} участниками конфликта являются не люди,
а различные психологические факторы внутреннего мира личности (потребности,
мотивы, ценности, чувства~и~т.~п.).

Внутриличностный конфликт может принимать различные формы. Одна из самых
распространенных --- ролевой конфликт, когда к одному человеку предъявляются
противоречивые требования по поводу того, каким должен быть результат его
работы.

\textbf{Межличностный конфликт} вовлекает двух или более индивидов,
воспринимающих себя как находящихся в оппозиции друг другу в отношении целей,
расположений, ценностей или поведения. Это самый распространенный тип конфликта.
Возможной причиной такого конфликта может быть несходство характеров, однако
чаще всего это борьба за ограниченные ресурсы, материальные средства и~т.~д. 

\textbf{Конфликт между личностью и группой} проявляется как противоречие между
ожиданиями или требованиями отдельной личности и сложившимися в группе нормами
поведения и труда. Такой конфликт может возникнуть, если эта личность займет
позицию, отличающуюся от позиции группы.

\textbf{Межгрупповые конфликты} --- конфликты внутри формальных групп,
коллектива, внутри неформальных групп, а также между формальными и неформальными
группами.

\section{По длительности протекания}

По длительности протекания конфликты могут быть~\cite{book08}:
\begin{itemize}
    \item кратковременными,
    \item затяжными.
\end{itemize}

\textbf{Кратковременные} чаще всего являются следствием взаимного непонимания или ошибок,
которые быстро осознаются. \textbf{Затяжные} связаны с глубокими
нравственно-психологическими травмами или с объективными трудностями.

Длительность конфликта зависит как от предмета противоречий, так и от характера
столкнувшихся людей. Длительные конфликты очень опасны, поскольку в них
конфликтующие личности закрепляют свое негативное состояние. Также частота
конфликтов может вызвать глубокую и длительную напряженность отношений.

\section{По источнику возникновения}

По источнику возникновения конфликты можно разделить на~\cite{book08}:
\begin{itemize}
    \item объективно обусловленные;
    \item субъективно обусловленные.
\end{itemize}

\textbf{Объективным} считается возникновение конфликта в сложной противоречивой
ситуации, в которой оказываются люди. Например, плохие условия труда, нечеткое
разделение функций и ответственности --- такого рода проблемы относятся к числу
конфликтных, т. е. объективно оказываются той почвой, на которой легко возникает
напряженная обстановка. Устранить конфликты, вызванные такими причинами, можно
только, изменив объективную ситуацию. В этих случаях конфликт выполняет своего
рода сигнальную функцию, указывая на неблагополучие в жизнедеятельности
коллектива.

\textbf{Субъективным} будет возникновение конфликта в связи с личностными
особенностями конфликтующих, с ситуациями, создающими преграды на пути
удовлетворения наших стремлений, желаний, интересов.

