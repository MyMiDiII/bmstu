\chapter*{ЗАКЛЮЧЕНИЕ}
\addcontentsline{toc}{chapter}{ЗАКЛЮЧЕНИЕ}

В данной работе было описано понятие конфликта, выявлена его строгая структура,
как статическая, так и динамическая, рассмотрены причины и механизм
возникновения конфликта, а также основные его типологии. Все это говорит о том,
что конфликт не является хаотичным и непредсказуемым элементом социальной жизни,
а подчиняется определенным правилам, уже хорошо изученным научным сообществом.
Следовательно, при необходимых знаниях конфликт можно не допускать, а при его
возникновении управлять им, увеличивая число конструктивных последствий и
предотвращая последствия деструктивные.

