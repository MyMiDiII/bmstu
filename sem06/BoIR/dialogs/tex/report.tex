\documentclass{bmstu}

% table of contents
\usepackage{tocloft}
\renewcommand{\cfttoctitlefont}{\hfill\large\bfseries}
\renewcommand{\cftaftertoctitle}{\hfill\hfill}
\renewcommand{\cftchapleader}{\cftdotfill{\cftdotsep}}
\setlength\cftbeforetoctitleskip{-22pt}
\setlength\cftaftertoctitleskip{15pt}

% searchable and copyable
\usepackage{cmap}

% Times New Roman
\usepackage{pscyr}
\renewcommand{\rmdefault}{ftm}

% Links
\def\UrlBreaks{\do\/\do-\do\_}
\usepackage[nottoc]{tocbibind} % for bib link
\usepackage[numbers]{natbib}
\renewcommand*{\bibnumfmt}[1]{#1.}

%% Images
\usepackage{graphicx}
\usepackage{wrapfig}

\newcommand{\img}[4] {
	\begin{figure}[ht!]
		\center{
            \includegraphics[height=#1]{../data/img/#2}
            \caption{#3}
            \label{img:#4}
        }
	\end{figure}
}

%% Captions
\usepackage[figurename=Рисунок,labelsep=endash,
            justification=centering]{caption}
\usepackage{hyphenat}

% Lists
\usepackage{enumitem}
\renewcommand{\labelitemi}{$-$}
\setlist{leftmargin=\parindent}

% Text
\usepackage{amstext}
\usepackage{ulem}
\renewcommand{\ULdepth}{1.5pt}

% Formulas
\DeclareMathOperator*{\argmax}{argmax}

% Tabulars
\usepackage{threeparttable}
\usepackage{csvsimple}
\usepackage{longtable,ltcaption,booktabs}
\usepackage{tabularx}
\usepackage{multirow}

% PDF
\usepackage{pdfpages}

% cites
\usepackage{natbib}
\bibpunct{[}{]}{;}{n}{}{}
%

% Listings
\usepackage{listings}
\usepackage[newfloat]{minted}
\usepackage{verbatim}

\include {listings.inc}

\begin{document}

\section*{<<Крепостная>>, 2021 год}

\begin{itemize}
    \item Не спешите, Петр Иванович.
    \item Катька? А ты что здесь делаешь?
    \item Решила вернуть себе свою собственность.
    \item Что?!
    \item Мой муж Вас обманул. Я не подписывала никаких документов.
        Дарственная не действительна. Моя подпись поддельна. Так что у Вас
        нет никаких прав на Червинку.
    \item ...
    \item Твари вы безродные, что ты, что муженек твой. Ни чести, ни
        совести.
    \item Вам ли говорить о чести, Петр Иванович, после всей той грязи,
        которую Вы напечатали в газетах, чтобы заполучить имение?
    \item Я лишь пытался вернуть то, что принадлежит мне. Свой дом.
        Свой, а не твой.
    \item Да, Червинка не мой дом и никогда им не была. Дом вещи там, где
        оставил ее хозяин, а я была вещью, Вашей с Анной Львовной вещью, и
        жить там ныне у меня нет никакого желания, однако и Вам принадлежать
        Червинка не будет.
    \item Что? Упиваешься возможностью отомстить хозяину? Да ты никто. Ты
        девка крепостная.
    \item Мне нет никакого дела до Вас, Петр Иванович. Ни мстить, ни объяснять
        я не хочу. Червинка не будет Вашей, но может принадлежать вашему сыну,
        но у меня есть два условия. Первое, Вы позволите Ларисе Викторовне быть
        рядом с сыном...
    \item Нет, эта женщина уже сделала свой выбор, и он был не в
        пользу меня и сына. Ноги её не будет рядом с нами.
    \item Значит, не будет и Червинки. Лев ещё совсем дитя, ему нужна
        материнская ласка.
    \item Ничего-ничего, я сам воспитаю его достойным нашей фамилии.
    \item Лучше просто любите его и позвольте Ларисе Викторовне делать то же
        самое. Ваша требовательность уже сгубила одного вашего сына, так не
        повторяйте ошибок.
    \item Ну и какое второе условие?
    \item В поместье никогда, никогда больше не будет крепостных. Работать
        будут только вольные и получать справедливую оплату за свой труд.
        Всем крепостным Вы дадите вольную. И больше никогда не будете покупать
        людей.
    \item Добродетельница! Всех наровишь осчастливить...
    \item Если Вы согласны, можем завтра же все оформить у натариуса.
    \item В 10 утра, в конторе. И попробуй только меня обмануть.
\end{itemize}

\section*{<<Два капитана>>, Виниамин Каверин, 1938-1944гг.}

– Саня, нам нужно поговорить об очень многих вещах, – сказал он серьезно. – И
мы, кажется, достаточно культурные люди, чтобы обсудить и решить все это мирным
путем, Не так ли?

Очевидно, он еще не забыл, как я однажды решил «все это» не очень мирным путем.
Но с каждым словом голос его становился тверже.

– Я не знаю, какие непосредственные причины побудили Катю внезапно уехать из
дому, но я вправе спросить: не связаны ли эти причины с твоим появлением?

– А ты бы спросил об этом у Кати, – отвечал я спокойно.

Он замолчал. У него запылали уши, а глаза вдруг стали бешеные, лоб разгладился.
Я смотрел на него с интересом.

– Однако мне известно, – начал он снова немного сдавленным голосом, – что она
уехала с тобою.

– Совершенно верно. Я даже помогал ей укладывать вещи.

– Так, – сказал он хрипло. Один глаз у него теперь был почти закрыт, а другим он
косил – довольно страшная картина. Таким я видел его впервые.

– Так, – снова повторил он.

– Да, так.

– Да.

– Мы помолчали.

– Послушай, – начал он снова. – Мы с тобой не договорили тогда на юбилее
Кораблева. Должен тебе сказать, что в общих чертах я знаю эту историю с
экспедицией «Святой Марии». Я тоже интересовался ею так же, как и ты, но,
пожалуй, с несколько иной точки зрения.

Я ничего не ответил. Мне была известна эта точка зрения.

– Между прочим, тебе, кажется, хотелось узнать, какую роль играл в этой
экспедиции Николай Антоныч. По крайней мере, так я мог судить по нашему
разговору.

Он мог судить об этом не только по нашему разговору. Но я не возражал ему. Я еще
не понимал, куда он клонит.

– Думаю, что могу оказать тебе в этом деле серьезную услугу.

– В самом деле?

– Да.

Он вдруг бросился ко мне, и я инстинктивно вскочил и стал за кресло.

– Послушай, послушай, – пробормотал он, – я знаю о нем такие вещи! Я знаю такую
штуку! У меня есть доказательства, от которых ему не поздоровится, если только
умеючи взяться за дело. Ты думаешь – он кто?

Три раза он повторил эту фразу, придвинувшись ко мне почти вплотную, так что мне
пришлось взять его за плечи и слегка отодвинуть. Но он этого даже не заметил.

– Такие штуки, о которых он сам забыл, – продолжал Ромашка. – В бумагах.

Конечно, он говорил о бумагах, взятых им у Вышимирского.

– Я знаю, отчего вы поссорились. Ты говорил, что он обокрал экспедицию, и он
тебя выгнал. Но это – правда. Ты оказался прав.

Второй раз я слышал это признание, но теперь оно доставило мне мало
удовольствия. Я только сказал с притворным изумлением:

– Да что ты?

– Это он! – с каким-то подлым упоением повторил Ромашка. – Я помогу тебе. Я тебе
все отдам, все доказательства. Он у нас полетит вверх ногами.

Нужно было промолчать, но я не удержался и спросил:

– За сколько?

Он опомнился.

– Ты можешь принять это как угодно, – сказал он. – Но я тебя прошу только об
одном: чтобы ты уехал.

– Один?

– Да.

– Без Кати?

– Да.

– Интересно. То есть, иными словами, ты просишь, чтобы я от нее отступился?

– Я люблю ее, – сказал он почти надменно.

– Ага, ты ее любишь! Это интересно. И чтобы мы не переписывались, не правда ли?

Он молчал.

– Подожди-ка минутку, я сейчас вернусь, – сказал я и вышел.

Саня позвал Николая Антоныча.

– Этот Ромашов, – продолжал я, – явился ко мне часа полтора тому назад и
предложил следующее: он предложил мне воспользоваться доказательствами, из
которых следует: во-первых, что вы обокрали экспедицию капитана Татаринова, а
во-вторых, еще разные штуки, касающиеся вашего прошлого, о которых вы не
упоминаете в анкетах.

Вот тут он уронил шляпу.

– У меня создалось впечатление, – продолжал я, – что этот товар он продает уже
не в первый раз. Не знаю, может быть, я ошибаюсь.

– Николай Антоныч! – вдруг закричал Ромашка. – Это все ложь. Не верьте ему. Он
врет.

Я подождал, пока он перестанет кричать.

– Конечно, теперь это, в сущности, все равно, – продолжал я, – теперь это дело
только ваших отношений. Но вы сознательно…

Я давно чувствовал, что на щеке прыгает какая-то жилка, и это мне не нравилось,
потому что я дал себе слово разговаривать с ними совершенно спокойно.

– Но вы сознательно шли на то, что этот человек может стать Катиным мужем. Вы
уговаривали ее – из подлости, конечно, – потому что вы его испугались. А теперь
он же приходит ко мне и кричит: «Он у нас полетит вверх ногами».

Как будто очнувшись, Николай Антоныч сделал шаг вперед и уставился на Ромашку.
Он смотрел на него долго, так долго, что даже и мне трудно было выдержать эту
напряженную тишину.

– Николай Антоныч, – снова жалостно пробормотал Ромашка.

Николай Антоныч все смотрел. Но вот он заговорил, и я поразился: у него был
надорванный, старческий голос.

– Зачем вы пригласили меня сюда? – спросил он. – Я болен, мне трудно говорить.
Вы хотели уверить меня, что он негодяй. Это для меня не новость. Вы хотели снова
уничтожить меня, но вы не в силах сделать больше того, что уже сделали – и
непоправимо. – Он глубоко вздохнул. Действительно, я видел, что говорить ему
было трудно.

– На ее суд, – продолжал он так же тихо, но уже с другим, ожесточенным
выражением, – отдаю я тот поступок, который она совершила, уйдя и не сказав мне
ни слова, поверив подлой клевете, которая преследует меня всю жизнь.

Я молчал. Ромашка дрожащей рукой налил стакан воды и поднес ему.

– Николай Антоныч, – пробормотал он, – вам нельзя волноваться.

Но Николай Антоныч с силой отвел его руку, и вода пролилась на ковер.

– Не принимаю, – сказал он и вдруг сорвал с себя очки и стал мять их в пальцах.
– Не принимаю ни упреков, ни сожаления. Ее дело. Ее личная судьба. А я одного ей
желал: счастья. Но память о брате я никому не отдам, – сказал он хрипло, и у
него стало угрюмое, одутловатое лицо с толстыми губами. – Я, может быть, рад был
бы поплатиться и этим страданием – уж пускай до смерти, потому что мне жизнь
давно не нужна. Но не было этого, и я отвергаю эти страшные, позорные обвинения.
И хоть не одного, а тысячу ложных свидетелей приведите, – все равно никто не
поверит, что я убил этого человека с его мыслями великими, с его великой душой.

Я хотел напомнить Николаю Антонычу, что он не всегда был такого высокого мнения
о своем брате, но он не дал мне заговорить.

– Только одного свидетеля я признаю, – продолжал он, – его самого, Ивана. Он
один может обвинить меня, и если бы я был виноват, он один бы имел на это право.

Николай Антоныч заплакал. Он порезал пальцы очками и стал долго вынимать носовой
платок. Ромашка подскочил и помог ему, но Николай Антоныч снова отстранил его
руки.

– Здесь бы и мертвый, кажется, заговорил, – сказал он и, болезненно, часто дыша
потянулся за шляпой.

– Николай Антоныч, – сказал я очень спокойно, – не думайте, что я намерен отдать
всю жизнь, чтобы убедить человечество в том, что вы виноваты. Для меня это давно
ясно, а теперь и не только для меня. Я пригласил вас не для этого разговора.
Просто я считал своим долгом раскрыть перед вами истинное лицо этого прохвоста.
Мне не нужно то, что он сообщил о вас, – больше того, я давно знаю все это.
Хотите ли вы сказать ему что-нибудь?

Николай Антоныч молчал.

– Ну, тогда пошел вон! – сказал я Ромашке.

Он бросился было к Николаю Антонычу и стал ему что-то шептать. Но, как
бесчувственный, стоял, глядя прямо перед собой, Николай Антоныч.

\section*{<<Доктор Кто>>}

\subsection*{13 Доктор}

— Я знаю, о чем ты попросишь. Но семейная история и путешествия во времени...
Сложно.

— Всего на часок. Посмотреть на нее издалека. Зачем нужен друг с машиной
времени, если нельзя слетать назад и увидеть бабушку молодой?

— Знаешь время и место?

— Знаю, что она жила в Лахоре в 50-е, но помимо этого...

— Я, конечно, могла бы, но... Не стоит. Если только... Нет, слишком
непредсказуемо.

— Что могла бы?

— Это риск.

— Будто все наши путешествия были без риска!

— Я же извинилась за армию «Смертоносных черепах». Как следует. Предположим, я
могла бы зациклить их в телепатические схемы ТАРДИС.

— Так и эта штука телепатическая?

— Не называй ее так, Грэм. И да, она обладает чем-то вроде телепатической
навигации. Если кратко об очень сложном процессе за пределами твоего понимания.

— Спасибо большое. Я торчу тут, чтобы меня оскорбляли.

— На любом объекте накапливаются фрагменты пространственно-временных частиц, на
протяжении жизни. ТАРДИС может считать их, как временные метки. Что думаете?

— Да, мне нравится. Пакистан. Никогда там не был. Минус один пункт в списке
желаний. Если только там нет черепах-убийц.

— Да, я только за.

— Один час, никакого...

— Вмешательства.

— Давай же, ты знаешь, что можешь. Встряхнемся.

\subsection*{11 Доктор}

— Полковник Мантон, прикажите своим людям отступить.

— Нет! Полковник Мантон, я хочу, чтобы вы велели своим людям бежать.

— Что?

— Это слово такое «бежать». Я хочу, чтобы вы стали известны этим словом. Я хочу,
чтобы вас называли «полковник-беглец». Я хочу, чтобы под вашей дверью смеялись
дети, потому что нашли дом «полковника-беглеца». И когда к вам придут и спросят:
«Хороша ли идея, добраться до меня… через дорогих мне людей…», я хочу, чтобы вы
назвали им свое имя. Надо же, я разъярен… что-то новенькое. Я действительно не
знаю, что сейчас будет…

— Гнев хорошего человека — не проблема. У хорошего человека слишком много
правил.

— Хорошим людям правила не нужны. Сегодня не тот день, чтобы проверять их
наличие.

\section*{Сколько детей было на Галлифрее}

— Вы когда-нибудь подсчитывали, сколько детей было на Галлифрее в тот день?

— Понятия не имею.

— Сколько тебе сейчас?

— Не знаю. Потерял счет. Тысяча двести с хвостиком, если не вру. Не помню даже
вру ли я насчет возраста – вот насколько я стар.

— На четыреста лет старше меня. И за всё это время не поинтересовался их числом?
Ни разу не подсчитал?

— Скажи, а в чем был бы смысл?

[10 Доктор] — Два миллиарда сорок семь миллионов...

— Ты посчитал!

— Ты забыл? Четыреста лет и ты уже забыл?!

— Я двигаюсь вперед...

— Куда? Где ты сейчас, если способен забыть такое?!

— Спойлеры...

— Нет, нет. На это раз я хочу знать, куда направляюсь.

— Нет, не хочешь!


\end{document}
